\documentclass[a4paper]{jsarticle}
\usepackage[dvips]{graphicx}
\usepackage{amsmath,amssymb,bm}
\usepackage{ascmac}
\usepackage{algorithm,algpseudocode}
\usepackage{svg}
\usepackage{url}

\flushbottom
\sloppy

\setlength{\paperwidth}{210mm}
\setlength{\paperheight}{297mm}
\setlength{\voffset}{0mm}
\setlength{\hoffset}{0mm}
\setlength{\textwidth}{\paperwidth}
\addtolength{\textwidth}{-30mm}
\setlength{\textheight}{\paperheight}
\addtolength{\textheight}{-60mm}
\setlength{\topmargin}{-1in}
\addtolength{\topmargin}{20mm}
\setlength{\headheight}{0mm}
\setlength{\headsep}{0mm}
\setlength{\footskip}{10mm}
\setlength{\oddsidemargin}{-1in}
\addtolength{\oddsidemargin}{15mm}
\setlength{\columnsep}{7mm}

\title{\bf 逐次添加される3次元点群の法線ベクトルを推定する3つのアイデア}
\author{\Large{\bf 杉原 知道}}
\date{}

\begin{document}
\maketitle
\vspace{-\baselineskip}

\section{与件}

点群$\mathcal{P}_{N}=\left\{\bm{p}_{i}\right\}$($i=1,\cdots,N$)の重心(barycenter)$\bar{\bm{p}}_{N}$と
分散共分散行列(variant-covariant matrix)$\bm{V}_{N}$,
さらに最小主成分$\lambda_{N3}$とそれに対する主軸方向$\bm{u}_{N3}$も求まっている下で,
新たに点$\bm{p}_{N+1}$が与えられたとする.
このとき,
点群$\mathcal{P}_{N+1}=\mathcal{P}_{N}\cup\left\{\bm{p}_{N+1}\right\}$の重心(barycenter)$\bar{\bm{p}}_{N+1}$と
分散共分散行列$\bm{V}_{N+1}$,
最小主成分$\lambda_{(N+1)3}$とそれに対する主軸方向$\bm{u}_{(N+1)3}$をなるべく効率良く求めたい.

少なくとも重心と分散共分散行列については
\begin{align*}
\bar{\boldsymbol{p}}_{N}=\frac{1}{N}\sum_{i=1}^{N}\boldsymbol{p}_{i}
\\
\boldsymbol{V}_{N}=\sum_{i=1}^{N}(\boldsymbol{p}_{i}-\bar{\boldsymbol{p}}_{N})(\boldsymbol{p}_{i}-\bar{\boldsymbol{p}}_{N})^{\mathrm{T}}
=\sum_{i=1}^{N}\boldsymbol{p}_{i}\boldsymbol{p}_{i}^{\mathrm{T}}-N\bar{\boldsymbol{p}}_{N}\bar{\boldsymbol{p}}_{N}^{\mathrm{T}}
\end{align*}
より,
\begin{align*}
\bar{\bm{V}}_{N}=\sum_{i=1}^{N}\boldsymbol{p}_{i}\boldsymbol{p}_{i}^{\mathrm{T}}
\end{align*}
とおくと,漸化式
\begin{align*}
\bar{\boldsymbol{p}}_{N+1}=\frac{1}{N+1}\left(N\bar{\boldsymbol{p}}_{N}+\boldsymbol{p}_{N+1}\right)
\\
\bar{\bm{V}}_{N+1}=\bar{\bm{V}}_{N}+\boldsymbol{p}_{N+1}\boldsymbol{p}_{N+1}^{\mathrm{T}}
\\
\bm{V}_{N+1}=\bar{\bm{V}}_{N+1}-(N+1)\bar{\bm{p}}_{N+1}\bar{\bm{p}}_{N+1}^{\mathrm{T}}
\end{align*}
から求まる.
あるいは
\begin{align*}
\bar{\bm{V}}_{N}=\bm{V}_{N}+N\bar{\bm{p}}_{N}\bar{\bm{p}}_{N}^{\mathrm{T}}
\end{align*}
より
\begin{align*}
\bm{V}_{N+1}=
\bm{V}_{N}+N\bar{\bm{p}}_{N}\bar{\bm{p}}_{N}^{\mathrm{T}}
+\boldsymbol{p}_{N+1}\boldsymbol{p}_{N+1}^{\mathrm{T}}
-(N+1)\bar{\bm{p}}_{N+1}\bar{\bm{p}}_{N+1}^{\mathrm{T}}
\end{align*}
としても良い.
いずれにしても,
残る$\lambda_{(N+1)3}$および$\bm{u}_{(N+1)3}$をいかに効率良く求めるかが問題である.


\section{アイデア1:直接解法}

3×3行列なので大した計算コストではない,と仮定し,定義通り$\bm{V}_{N+1}$の
固有値$\lambda_{(N+1)i}$($i=1,2,3$)およびそれらに対する固有ベクトル$\bm{u}_{(N+1)i}$($i=1,2,3$)を直接計算する.


\section{アイデア2:べき乗法}

$\bm{u}_{N3}$を$\bm{u}_{(N+1)3}$の初期推定値として,
べき乗法により$\lambda_{(N+1)3}$および$\bm{u}_{(N+1)3}$を求める.
$\lambda_{(N+1)3}$は最小固有値なので,$\bm{V}_{N+1}^{-1}$を求めて逆反復法を適用する.

\section{アイデア3:べき乗法の逆行列計算にSherman-Morrison公式を用いる}

$\bm{V}_{N}^{-1}$は既に求まっているとして,
Sherman-Morrisonの公式より
\begin{align*}
\bm{V}_{N+1}^{-1}
=(\bm{V}_{N}+\bm{p}_{N+1}\bm{p}_{N+1}^{\mathrm{T}})^{-1}
=\bm{V}_{N}^{-1}-\frac{\bm{V}_{N}^{-1}\bm{p}_{N+1}\bm{p}_{N+1}^{\mathrm{T}}\bm{V}_{N}^{-1}}{1+\bm{p}_{N+1}^{\mathrm{T}}\bm{V}_{N}^{-1}\bm{p}_{N+1}}
\end{align*}



\section{法線をとる点が決まっている場合}

点群$\mathcal{P}_{N}$の,重心ではなく与えられたある点$\bm{p}_{0}$を中心とする
分散共分散行列$\bm{V}_{0N}$が求まっている下で,
新たに点$\bm{p}_{N+1}$が与えられたとする.
このとき
\begin{align*}
\boldsymbol{V}_{N}=\sum_{i=1}^{N}(\boldsymbol{p}_{i}-\bar{\boldsymbol{p}}_{0})(\boldsymbol{p}_{i}-\bar{\boldsymbol{p}}_{0})^{\mathrm{T}}
\end{align*}
より,漸化式
\begin{align*}
\bm{V}_{0(N+1)}=\bm{V}_{0N}+(\boldsymbol{p}_{N+1}-\bm{p}_{0})(\boldsymbol{p}_{N+1}-\bm{p}_{0})^{\mathrm{T}}
\end{align*}
を得る.






\end{document}
