\documentclass[a4paper]{jsarticle}
\usepackage{amsmath,amssymb,bm}
\usepackage{graphicx}
\usepackage{ascmac}
\everymath{\displaystyle}

\flushbottom
\sloppy

\setlength{\paperwidth}{210mm}
\setlength{\hoffset}{0mm}
\setlength{\textwidth}{\paperwidth}
\addtolength{\textwidth}{-30mm}
\setlength{\oddsidemargin}{-1in}
\addtolength{\oddsidemargin}{15mm}
\setlength{\columnsep}{7mm}

\setlength{\paperheight}{297mm}
\setlength{\voffset}{0mm}
\setlength{\textheight}{\paperheight}
\addtolength{\textheight}{-60mm}
\setlength{\topmargin}{-1in}
\addtolength{\topmargin}{20mm}
\setlength{\headheight}{0mm}
\setlength{\headsep}{0mm}
\setlength{\footskip}{10mm}

\title{非対称正方行列の固有値・固有ベクトル計算}

\begin{document}
\maketitle

\section{固有値・固有ベクトル計算の流れ}

Hessenberg行列への変換

右三角行列への変換

\section{Hessenberg行列への変換}


元の行列を
\begin{align*}
\bm{M}=\begin{bmatrix}
m_{11} & m_{12} & \cdots & m_{1m} \\
m_{21} & m_{22} & \cdots & m_{2m} \\
\vdots & \vdots & \ddots & \vdots \\
m_{n1} & m_{n2} & \cdots & m_{nm} \\
\end{bmatrix}
\end{align*}
とおく.



\section{右三角行列への変換}


\end{document}
