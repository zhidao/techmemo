\documentclass[a4paper]{jsarticle}
\usepackage[dvips]{graphicx}
\usepackage{amsmath,amssymb,bm}
\usepackage{ascmac}
\usepackage{algorithm,algpseudocode}
\usepackage{svg}
\usepackage{url}

\flushbottom
\sloppy

\setlength{\paperwidth}{210mm}
\setlength{\paperheight}{297mm}
\setlength{\voffset}{0mm}
\setlength{\hoffset}{0mm}
\setlength{\textwidth}{\paperwidth}
\addtolength{\textwidth}{-30mm}
\setlength{\textheight}{\paperheight}
\addtolength{\textheight}{-60mm}
\setlength{\topmargin}{-1in}
\addtolength{\topmargin}{20mm}
\setlength{\headheight}{0mm}
\setlength{\headsep}{0mm}
\setlength{\footskip}{10mm}
\setlength{\oddsidemargin}{-1in}
\addtolength{\oddsidemargin}{15mm}
\setlength{\columnsep}{7mm}

\title{\bf サポートベクターマシン}
\author{\Large{\bf 杉原 知道}}
\date{}

\begin{document}
\maketitle
\vspace{-\baselineskip}

\section{基本的な考え方}

サポートベクターマシン(Support Vector Machine, SVM)の自分なりの解説.

$n$次元空間$\mathbb{R}^{n}$が,
ベクトル$\bm{\nu}$および定数$c$で規定される
ある超平面$\pi=\left\{\bm{p}\left|\bm{\nu}^{\mathrm{T}}\bm{p}+c=0\right.\right\}$によって,
二つの領域$\mathcal{A}$,$\mathcal{B}$に分離される状況を考えます
(ややこしいので,$\pi$自身は$\mathcal{A}$と$\mathcal{B}$のどちらにも含まれないものとします).
\begin{align*}
\mathcal{A}=\left\{\bm{p}\left|d(\bm{p})>0\right.\right\}
\\
\mathcal{B}=\left\{\bm{p}\left|d(\bm{p})<0\right.\right\}
%% \\
%% \mathcal{A}\cup\mathcal{B}\cup\pi=\mathbb{R}^{n},\qquad
%% \mathcal{A}\cap\mathcal{B}=\emptyset,\qquad
%% \mathcal{A}\cap\pi=\emptyset,\qquad
%% \mathcal{B}\cap\pi=\emptyset
\end{align*}
ただし,任意の点$\bm{p}\in\mathbb{R}^{n}$に対し
\begin{align*}
d(\bm{p})=\bm{\nu}^{\mathrm{T}}\bm{p}+c
\end{align*}
とおきました.
$\bm{\nu}$は$\pi$の法線ベクトルに相当します.
ノルムが不定ですが,ここでは$\|\bm{\nu}\|=1$(すなわち単位ベクトル)としましょう.
このとき,$d(\bm{p})$は点$\bm{p}$から平面$\pi$までの符号付き距離を表していることになります.

観測によって
$\mathcal{A}$の標本群$\hat{\mathcal{P}}_{\mathrm{A}}=\left\{\hat{\bm{p}}_{i}\left|\forall\hat{\bm{p}}_{i}\in\mathcal{A}\right.\right\}$と
$\mathcal{B}$の標本群$\hat{\mathcal{P}}_{\mathrm{B}}=\left\{\hat{\bm{p}}_{i}\left|\forall\hat{\bm{p}}_{i}\in\mathcal{B}\right.\right\}$が
得られたとき,超平面$\pi$を逆に同定したい,というのがここでの問題設定です.

多くの場合,$\hat{\pi}$の候補は無数に存在します.
\begin{figure}[h]
\centering
\includesvg[height=.3\textheight]{fig/linearly_separable_points.svg}
\end{figure}
そのような中から「最もよく$\hat{\mathcal{P}}_{\mathrm{A}}$と$\hat{\mathcal{P}}_{\mathrm{B}}$を分離する」超平面を選ぶことにしましょう.
最もよく分離するとは,
$\hat{\bm{p}}_{i}\in\hat{\mathcal{P}}_{\mathrm{A}}$のうち$\hat{\pi}$に最も近いものを$\hat{\bm{p}}_{\mathrm{A}}^{*}$,
$\hat{\bm{p}}_{i}\in\hat{\mathcal{P}}_{\mathrm{B}}$のうち$\hat{\pi}$に最も近いものを$\hat{\bm{p}}_{\mathrm{B}}^{*}$
(これらを{\bf サポートベクター}と呼びます),
点$\bm{p}$から平面$\hat{\pi}$までの符号付き距離を
\begin{align*}
\hat{d}(\bm{p})=\hat{\bm{\nu}}^{\mathrm{T}}\bm{p}+\hat{c}
\end{align*}
とそれぞれおいたとき,
$\hat{d}(\hat{\bm{p}}_{\mathrm{A}}^{*})-\hat{d}(\hat{\bm{p}}_{\mathrm{B}}^{*})$が最大となる,
ということを意味するものとします.
このような超平面もまた無数に存在しますが,
$\hat{d}(\hat{\bm{p}}_{\mathrm{A}}^{*})=-\hat{d}(\hat{\bm{p}}_{\mathrm{B}}^{*})$となるものを選択するのが自然でしょう.
\begin{figure}[h]
\centering
\includesvg[height=.3\textheight]{fig/support_plane.svg}
\end{figure}

$\hat{\bm{p}}_{\mathrm{A}}^{*}$,$\hat{\bm{p}}_{\mathrm{B}}^{*}$を見つけるのは容易ではありませんが,
$\hat{\pi}$は次の最適化問題を解くことによって同定できます.
\begin{align*}
(\hat{\bm{\nu}},\hat{c})=\mathop{\mathrm{arg~max}}_{(\bm{\nu},c)}\left\{
\bar{d}\left|
d(\bm{p}_{i})\geq \bar{d}\quad\mbox{for}\quad\forall\bm{p}_{i}\in\hat{\mathcal{P}}_{\mathrm{A}},\quad
d(\bm{p}_{i})\leq-\bar{d}\quad\mbox{for}\quad\forall\bm{p}_{i}\in\hat{\mathcal{P}}_{\mathrm{B}},\quad
\|\bm{\nu}\|=1
\right.\right\}
\end{align*}
$\bar{d}$が$\hat{d}(\hat{\bm{p}}_{\mathrm{A}}^{*})=-\hat{d}(\hat{\bm{p}}_{\mathrm{B}}^{*})$に相当します.

ここで,
$\bm{\nu}^{\prime}=\bm{\nu}/\bar{d}$,
$c^{\prime}=c/\bar{d}$とそれぞれおきましょう.
明らかに$\bar{d}>0$ですので,
$\bar{d}=1/\|\bm{\nu}^{\prime}\|$であることに注意すれば,
上記の問題は次のように変形できます.
\begin{align*}
(\hat{\bm{\nu}}^{\prime},\hat{c}^{\prime})
=\mathop{\mathrm{arg~max}}_{(\bm{\nu}^{\prime},c^{\prime})}\left\{
\left.\frac{1}{\|\bm{\nu}^{\prime}\|}\right|
\bm{\nu}^{\prime\mathrm{T}}\bm{p}_{i}+c^{\prime}\geq 1\quad\mbox{for}\quad\forall\bm{p}_{i}\in\hat{\mathcal{P}}_{\mathrm{A}},\quad
\bm{\nu}^{\prime\mathrm{T}}\bm{p}_{i}+c^{\prime}\leq-1\quad\mbox{for}\quad\forall\bm{p}_{i}\in\hat{\mathcal{P}}_{\mathrm{B}}
\right\}
\\
=\mathop{\mathrm{arg~min}}_{(\bm{\nu}^{\prime},c^{\prime})}\left\{
\left.\frac{1}{2}\bm{\nu}^{\prime\mathrm{T}}\bm{\nu}^{\prime}\right|
\bm{p}_{i}^{\mathrm{T}}\bm{\nu}^{\prime}+c^{\prime}\geq 1\quad\mbox{for}\quad\forall\bm{p}_{i}\in\hat{\mathcal{P}}_{\mathrm{A}},\quad
-\bm{p}_{i}^{\mathrm{T}}\bm{\nu}^{\prime}-c^{\prime}\geq 1\quad\mbox{for}\quad\forall\bm{p}_{i}\in\hat{\mathcal{P}}_{\mathrm{B}}
\right\}
\end{align*}
これは二次計画問題ですので,有効制約法などを使って解くことが出来ます.
$(\hat{\bm{\nu}},\hat{c})$は,
得られた$(\hat{\bm{\nu}}^{\prime},\hat{c}^{\prime})$から
$\bm{\nu}=\bm{\nu}^{\prime}/\|\bm{\nu}^{\prime}\|$,
$c=c^{\prime}/\|\bm{\nu}^{\prime}\|$とそれぞれ求まります.

サポートベクターマシンとは,同定された超平面$\hat{\pi}=\left\{\bm{p}\left|\hat{\bm{\nu}}^{\mathrm{T}}\bm{p}+\hat{c}=0\right.\right\}$を用いて構成される識別器のことです.
すなわち新たに得られた点が
$\hat{d}(\bm{p})>0$を満たすならば$\bm{p}\in\mathcal{A}$,
$\hat{d}(\bm{p})<0$を満たすならば$\bm{p}\in\mathcal{B}$と判定できるわけです.


\section{実用のためのロバスト化}

実際のデータには誤差が載るため,ときどき境界となる超平面を超えて反対側の領域に観測されてしまうものもあります.
境界付近は下図のように渾然として,綺麗に分離できないのが普通です.

\begin{figure}[h]
\centering
\includesvg[height=.3\textheight]{fig/support_plane_outlier.svg}
\end{figure}

この場合,前節の計算をそのまま適用すると「解なし」という結果になってしまいます.

$\hat{\mathcal{P}}_{\mathrm{A}}$の全ての点$\bm{p}_{i}$について$\hat{d}(\bm{p}_{i})$を$\eta_{i}$,
同様に$\hat{\mathcal{P}}_{\mathrm{B}}$の全ての点$\bm{p}_{i}$にて$\hat{d}(\bm{p}_{i})$を$-\eta_{i}$
それぞれずらすことで,このような渾然とした状況が解消されるとしましょう.
ただし,全ての$\eta_{i}$は非負であるとします.
このとき,超平面を規定する$\hat{\bm{\nu}}$,$\hat{c}$と,
新たに加わった未知数$\{\eta_{i}\}$は,次の最適化問題の解となるでしょう.

\begin{align*}
(\hat{\bm{\nu}},\hat{c},\{\eta_{i}\})=\mathop{\mathrm{arg~max}}_{(\bm{\nu},c,\{\eta_{i}\})}\left\{
\bar{d}\left|
d(\bm{p}_{i})+\eta_{i}\geq \bar{d}\quad\mbox{for}\quad\forall\bm{p}_{i}\in\hat{\mathcal{P}}_{\mathrm{A}},\quad
d(\bm{p}_{i})-\eta_{i}\leq-\bar{d}\quad\mbox{for}\quad\forall\bm{p}_{i}\in\hat{\mathcal{P}}_{\mathrm{B}},\quad
\|\bm{\nu}\|=1
\right.\right\}
\end{align*}

前節と同様に不等式の両辺を$\bar{d}$で割ったのち,$\eta_{i}/\bar{d}$を改めて$\eta_{i}$とおくと,
上記の問題は次のように変形できます.

\begin{align*}
(\hat{\bm{\nu}}^{\prime},\hat{c}^{\prime},\{\eta_{i}\})
=\mathop{\mathrm{arg~min}}_{(\bm{\nu}^{\prime},c^{\prime},\{\eta_{i}\})}\left\{
\left.\frac{1}{2}\bm{\nu}^{\prime\mathrm{T}}\bm{\nu}^{\prime}\right|
\bm{p}_{i}^{\mathrm{T}}\bm{\nu}^{\prime}+c^{\prime}+\eta_{i}\geq 1\quad\mbox{for}\quad\forall\bm{p}_{i}\in\hat{\mathcal{P}}_{\mathrm{A}},\quad
-\bm{p}_{i}^{\mathrm{T}}\bm{\nu}^{\prime}-c^{\prime}-\eta_{i}\geq 1\quad\mbox{for}\quad\forall\bm{p}_{i}\in\hat{\mathcal{P}}_{\mathrm{B}}
\right\}
\end{align*}




\end{document}
