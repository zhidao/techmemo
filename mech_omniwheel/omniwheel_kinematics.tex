\documentclass[a4paper]{jsarticle}
\usepackage[dvips]{graphicx}
\usepackage{svg}
\usepackage{amsmath,amssymb,bm}

\flushbottom
\sloppy

\setlength{\paperwidth}{210mm}
\setlength{\paperheight}{297mm}
\setlength{\voffset}{0mm}
\setlength{\hoffset}{0mm}
\setlength{\textwidth}{\paperwidth}
\addtolength{\textwidth}{-30mm}
\setlength{\textheight}{\paperheight}
\addtolength{\textheight}{-40mm}
\setlength{\topmargin}{-1in}
\addtolength{\topmargin}{20mm}
\setlength{\headheight}{0mm}
\setlength{\headsep}{0mm}
\setlength{\footskip}{10mm}
\setlength{\oddsidemargin}{-1in}
\addtolength{\oddsidemargin}{15mm}
\setlength{\columnsep}{7mm}

\everymath{\displaystyle}

\title{\bf {\LARGE 球状全方位車輪の運動学/力学}}
\author{杉原知道}

\begin{document}

%\section{運動学}

\begin{figure}[h]
\begin{center}
\includesvg[height=.3\textheight]{omniball_kinematics.svg}
\label{fig:omniball_kinematics}
\caption{球状全方位車輪の運動学}
\end{center}
\end{figure}

台車座標系における球状車輪$i$($i=1,2,3,4$)の水平面内中心位置ベクトルを$\bm{s}_{i}$,
駆動輪および受動輪の回転軸方向単位ベクトルを$\hat{\bm{z}}_{\mathrm{D}i}$,$\hat{\bm{z}}_{\mathrm{P}i}$とそれぞれおく.
配置および各動輪の回転正方向が図\ref{fig:omniball_kinematics}のようであるならば,
\begin{align*}
& \bm{s}_{1}=\begin{bmatrix} R \\ -R \end{bmatrix}, &
& \hat{\bm{z}}_{\mathrm{D}1}=\frac{1}{\sqrt{2}}\begin{bmatrix}-1 \\ 1 \end{bmatrix}, &
& \hat{\bm{z}}_{\mathrm{P}1}=\frac{1}{\sqrt{2}}\begin{bmatrix} 1 \\ 1 \end{bmatrix}
\\
& \bm{s}_{2}=\begin{bmatrix} R \\ R \end{bmatrix}, &
& \hat{\bm{z}}_{\mathrm{D}2}=\frac{1}{\sqrt{2}}\begin{bmatrix}-1 \\ -1 \end{bmatrix}, &
& \hat{\bm{z}}_{\mathrm{P}2}=\frac{1}{\sqrt{2}}\begin{bmatrix}-1 \\ 1 \end{bmatrix},
\\
& \bm{s}_{3}=\begin{bmatrix}-R \\ R \end{bmatrix}, &
& \hat{\bm{z}}_{\mathrm{D}3}=\frac{1}{\sqrt{2}}\begin{bmatrix} 1 \\ -1 \end{bmatrix}, &
& \hat{\bm{z}}_{\mathrm{P}3}=\frac{1}{\sqrt{2}}\begin{bmatrix}-1 \\ -1 \end{bmatrix},
\\
& \bm{s}_{4}=\begin{bmatrix}-R \\ -R \end{bmatrix}, &
& \hat{\bm{z}}_{\mathrm{D}4}=\frac{1}{\sqrt{2}}\begin{bmatrix} 1 \\ 1 \end{bmatrix}, &
& \hat{\bm{z}}_{\mathrm{P}4}=\frac{1}{\sqrt{2}}\begin{bmatrix} 1 \\ -1 \end{bmatrix},
\end{align*}
である.
以下はこの配置を仮定する.

球状車輪の半径を$r$,
慣性系における台車座標系の並進速度を$\bm{v}_{\mathrm{B}}=[v_{\mathrm{B}x}~v_{\mathrm{B}y}]^{\mathrm{T}}$,
旋回角速度を$\omega_{\mathrm{B}}$,
車輪$i$の駆動輪回転速度を$\omega_{\mathrm{D}i}$,
受動輪回転速度を$\omega_{\mathrm{P}i}$とそれぞれおこう.
台車座標系の方位角は常に慣性系と一致しているとしても一般性を失わない.
このとき,車輪$i$の水平面内中心(接触点)の速度$\bm{v}_{i}$は
\begin{align*}
\bm{v}_{i}
=\bm{v}_{\mathrm{B}}
+\omega_{\mathrm{B}}\bm{s}_{i}^{\times}
+r\omega_{\mathrm{D}i}\hat{\bm{z}}_{\mathrm{D}i}^{\times}
+r\omega_{\mathrm{P}i}\hat{\bm{z}}_{\mathrm{P}i}^{\times}
\end{align*}
となる.
ただし,任意の2次元ベクトル$\bm{a}=[a_{1}~a_{2}]^{\mathrm{T}}$に対し$\bm{a}^{\times}$を
\begin{align*}
\bm{a}^{\times}\overset{\mathrm{def}}{=}\begin{bmatrix} -a_{2} \\ a_{1} \end{bmatrix}
\end{align*}
と定義する.
全ての車輪が地面に対し滑りなく転がると仮定すると,
$\bm{v}_{i}=\bm{0}$であるから,
\begin{align*}
r\begin{bmatrix}
 \hat{\bm{z}}_{\mathrm{D}i}^{\times} & \hat{\bm{z}}_{\mathrm{P}i}^{\times}
\end{bmatrix}
\begin{bmatrix}
 \omega_{\mathrm{D}i} \\ \omega_{\mathrm{P}i}
\end{bmatrix}
=
-\begin{bmatrix}
 \bm{1} & \bm{s}_{i}^{\times}
\end{bmatrix}
\begin{bmatrix}
 \bm{v}_{\mathrm{B}} \\ \omega_{\mathrm{B}}
\end{bmatrix}
\quad\Leftrightarrow\quad
\begin{bmatrix}
 \omega_{\mathrm{D}i} \\ \omega_{\mathrm{P}i}
\end{bmatrix}
=
-\frac{1}{r}\begin{bmatrix}
 \hat{\bm{z}}_{\mathrm{D}i}^{\times} & \hat{\bm{z}}_{\mathrm{P}i}^{\times}
\end{bmatrix}^{-1}
\begin{bmatrix}
 \bm{1} & \bm{s}_{i}^{\times}
\end{bmatrix}
\begin{bmatrix}
 \bm{v}_{\mathrm{B}} \\ \omega_{\mathrm{B}}
\end{bmatrix}
\end{align*}
が成り立つ.
特に上記の配置の場合,
\begin{align*}
\hat{\bm{z}}_{\mathrm{D}i}^{\mathrm{T}}\hat{\bm{z}}_{\mathrm{D}i}^{\times}&=0, &
\hat{\bm{z}}_{\mathrm{D}i}^{\mathrm{T}}\hat{\bm{z}}_{\mathrm{P}i}^{\times}&=1,
\\
\hat{\bm{z}}_{\mathrm{P}i}^{\mathrm{T}}\hat{\bm{z}}_{\mathrm{D}i}^{\times}&=-1, &
\hat{\bm{z}}_{\mathrm{P}i}^{\mathrm{T}}\hat{\bm{z}}_{\mathrm{P}i}^{\times}&=0,
\\
\hat{\bm{z}}_{\mathrm{D}i}^{\mathrm{T}}\bm{s}_{i}^{\times}&=0, &
\hat{\bm{z}}_{\mathrm{P}i}^{\mathrm{T}}\bm{s}_{i}^{\times}&=\sqrt{2}R
\end{align*}
であるので,
\begin{align*}
\begin{bmatrix}
 \omega_{\mathrm{D}i} \\ \omega_{\mathrm{P}i}
\end{bmatrix}
=
-\frac{1}{r}
\begin{bmatrix}
-\hat{\bm{z}}_{\mathrm{P}i}^{\mathrm{T}} \\
 \hat{\bm{z}}_{\mathrm{D}i}^{\mathrm{T}}
\end{bmatrix}
\begin{bmatrix}
 \bm{1} & \bm{s}_{i}^{\times}
\end{bmatrix}
\begin{bmatrix}
 \bm{v}_{\mathrm{B}} \\ \omega_{\mathrm{B}}
\end{bmatrix}
=
\frac{1}{r}
\begin{bmatrix}
 \hat{\bm{z}}_{\mathrm{P}i}^{\mathrm{T}} & \sqrt{2}R \\
-\hat{\bm{z}}_{\mathrm{D}i}^{\mathrm{T}} & 0
\end{bmatrix}
\begin{bmatrix}
 \bm{v}_{\mathrm{B}} \\ \omega_{\mathrm{B}}
\end{bmatrix}
\end{align*}
が言える.

仮定した各ベクトルを実際に代入すると,
\begin{align*}
\begin{bmatrix}
 \omega_{\mathrm{D}1} \\ \omega_{\mathrm{P}1} \\
 \omega_{\mathrm{D}2} \\ \omega_{\mathrm{P}2} \\
 \omega_{\mathrm{D}3} \\ \omega_{\mathrm{P}3} \\
 \omega_{\mathrm{D}4} \\ \omega_{\mathrm{P}4}
\end{bmatrix}
=
\frac{1}{\sqrt{2}r}
\begin{bmatrix}
 1 &  1 & 2R \\
 1 & -1 &  0 \\
-1 &  1 & 2R \\
 1 &  1 &  0 \\
-1 & -1 & 2R \\
-1 &  1 &  0 \\
 1 & -1 & 2R \\
-1 & -1 &  0
\end{bmatrix}
\begin{bmatrix}
 v_{\mathrm{B}x} \\ v_{\mathrm{B}y} \\ \omega_{\mathrm{B}}
\end{bmatrix}
\end{align*}
を得る.
駆動輪回転速度のみを抜き出せば,
\begin{align*}
\begin{bmatrix}
 \omega_{\mathrm{D}1} \\
 \omega_{\mathrm{D}2} \\
 \omega_{\mathrm{D}3} \\
 \omega_{\mathrm{D}4}
\end{bmatrix}
=
\frac{1}{\sqrt{2}r}
\begin{bmatrix}
 1 &  1 & 2R \\
-1 &  1 & 2R \\
-1 & -1 & 2R \\
 1 & -1 & 2R \\
\end{bmatrix}
\begin{bmatrix}
 v_{\mathrm{B}x} \\ v_{\mathrm{B}y} \\ \omega_{\mathrm{B}}
\end{bmatrix}
\end{align*}
となる.
これは,
所望の台車並進・旋回速度を得るための速度逆運動学を与える.

\bigskip
%\section{等価駆動トルク}

車輪$i$の駆動輪に駆動トルク$\tau_{i}$を発生させたとき,
理想的に全ての車輪が地面に対し滑りなく転がり,
接触力によって台車に並進力$\bm{f}_{\mathrm{B}}$および旋回トルク$\tau_{\mathrm{B}}$を発生させたとすると,
仮想仕事の原理より
\begin{align*}
\bm{v}_{\mathrm{B}}^{\mathrm{T}}\bm{f}_{\mathrm{B}}+\omega_{\mathrm{B}}\tau_{\mathrm{B}}
\equiv\sum_{i=1}^{4}\omega_{\mathrm{D}i}\tau_{\mathrm{D}i}
\quad\Leftrightarrow\quad
\begin{bmatrix}
 \bm{v}_{\mathrm{B}}^{\mathrm{T}} & \omega_{\mathrm{B}}
\end{bmatrix}
\begin{bmatrix}
 \bm{f}_{\mathrm{B}} \\ \tau_{\mathrm{B}}
\end{bmatrix}
\equiv
\begin{bmatrix}
 \bm{v}_{\mathrm{B}}^{\mathrm{T}} & \omega_{\mathrm{B}}
\end{bmatrix}
\cdot
\frac{1}{\sqrt{2}r}
\begin{bmatrix}
 1 & -1 & -1 &  1 \\
 1 &  1 & -1 & -1 \\
2R & 2R & 2R & 2R
\end{bmatrix}
\begin{bmatrix}
 \tau_{\mathrm{D}1} \\
 \tau_{\mathrm{D}2} \\
 \tau_{\mathrm{D}3} \\
 \tau_{\mathrm{D}4}
\end{bmatrix}
\end{align*}
が任意の$\bm{v}_{\mathrm{B}}$,$\omega_{\mathrm{B}}$に対し恒等的に成り立つ.
よって
\begin{align*}
\begin{bmatrix}
 \bm{f}_{\mathrm{B}} \\ \tau_{\mathrm{B}}
\end{bmatrix}
=
\frac{1}{\sqrt{2}r}
\begin{bmatrix}
 1 & -1 & -1 &  1 \\
 1 &  1 & -1 & -1 \\
2R & 2R & 2R & 2R
\end{bmatrix}
\begin{bmatrix}
 \tau_{\mathrm{D}1} \\
 \tau_{\mathrm{D}2} \\
 \tau_{\mathrm{D}3} \\
 \tau_{\mathrm{D}4}
\end{bmatrix}
\end{align*}
が言える.



\end{document}
