\documentclass[a4paper]{jsarticle}
\usepackage[dvips]{graphicx}
\usepackage{svg}
\usepackage{amsmath,amssymb,bm}
\usepackage{algorithm,algpseudocode}

\flushbottom
\sloppy

%%% paper size = A4
\setlength{\paperwidth}{210mm}
\setlength{\paperheight}{297mm}

%%% truncate offset
\setlength{\voffset}{0mm}
\setlength{\hoffset}{0mm}

%%% text bounding size
\setlength{\textwidth}{\paperwidth}
\addtolength{\textwidth}{-30mm}
\setlength{\textheight}{\paperheight}
\addtolength{\textheight}{-60mm}

%%% text margin
\setlength{\topmargin}{-1in}
\addtolength{\topmargin}{20mm}
\setlength{\headheight}{0mm}
\setlength{\headsep}{0mm}
\setlength{\footskip}{10mm}

%\setlength{\oddsidemargin}{-30pt}
\setlength{\oddsidemargin}{-1in}
\addtolength{\oddsidemargin}{15mm}
\setlength{\columnsep}{7mm}

\everymath{\displaystyle}

\title{\bf Zeoの幾何要素(直線・線分・平面・三角形)基本計算}
\author{\Large{\bf 杉原 知道}}
\date{}

\begin{document}
\maketitle
\vspace{-\baselineskip}

\begin{figure}[h]
\begin{center}
\includesvg[height=.3\textheight]{fig/line_edge_plane_tri.svg}
\label{fig:line_edge_plane_tri}
\caption{直線・線分・平面・三角形と点の関係}
\end{center}
\end{figure}

\section{直線}

直線$\mathrm{L}$は,その直線が通るある一点の位置ベクトル$\bm{p}_{0}$と単位方向ベクトル$\bm{d}$で表します($\mathrm{L}\{\bm{p}_{0},\bm{d}\}$).
次のメソッドが用意されています.

\noindent{\tt zLine2D/3DCreateTwoPoints}
{\bf 2点$\bm{p}_{1}$,$\bm{p}_{2}$を通る直線$\mathrm{L}\left<\bm{p}_{1},\bm{p}_{2}\right>$} 

$\mathrm{L}\left<\bm{p}_{1},\bm{p}_{2}\right>
=\mathrm{L}\left\{\bm{p}_{1},\frac{\bm{p}_{2}-\bm{p}_{1}}{\|\bm{p}_{2}-\bm{p}_{1}\|}\right\}$

\noindent{\tt zLine2D/3DPoint}
{\bf $\mathrm{L}$上の点$\bm{p}(k;\mathrm{L})$(パラメータ$k$)}

$\bm{p}(k;\mathrm{L})=\bm{p}_{0}+k\bm{d}$

\noindent{\tt zLine2D/3DProjPoint}
{\bf ある点$\bm{p}$の$\mathrm{L}$への射影$\bm{p}_{\mathrm{P}}(\bm{p};\mathrm{L})$}

$\bm{p}_{\mathrm{P}}(\bm{p};\mathrm{L})=\bm{p}(\bm{d}^{\mathrm{T}}\varDelta\bm{p};\mathrm{L})$
(ただし,$\varDelta\bm{p}\overset{\mathrm{def}}{=}\bm{p}-\bm{p}_{0}$)

\noindent{\tt zLine2D/3DDistFromPoint}
{\bf ある点$\bm{p}$と$\mathrm{L}$との距離$d(\bm{p};\mathrm{L})$}

$d(\bm{p};\mathrm{L})=\|\varDelta\bm{p}\times\bm{d}\|$

\noindent{\tt zLine2D/3DPointIsOn}
{\bf ある点$\bm{p}$が$\mathrm{L}$上に載っているか否か(マージン$d_{\mathrm{m}}$)}

$d(\bm{p};\mathrm{L})\leq d_{\mathrm{m}}$

\noindent{\tt zIntersectLine2D}
{\bf 2直線$\mathrm{L}_{1}\{\bm{p}_{10},\bm{d}_{1}\}$,$\mathrm{L}_{2}\{\bm{p}_{20},\bm{d}_{2}\}$の交点$\bm{p}_{\mathrm{X}}(\mathrm{L}_{1},\mathrm{L}_{2})$}(2Dのみ)

$\bm{p}_{\mathrm{X}}(\mathrm{L}_{1},\mathrm{L}_{2})=\bm{p}(\bm{p}_{10}+s\bm{d}_{1})$
(ただし,$[s~~t]^{\mathrm{T}}=\left[\bm{d}_{1}~~-\bm{d}_{2}\right]^{-1}(\bm{p}_{20}-\bm{p}_{10})$)

\noindent{\tt zLine3DCommonPerpEnd}
{\bf 2直線$\mathrm{L}_{1}\{\bm{p}_{10},\bm{d}_{1}\}$,$\mathrm{L}_{2}\{\bm{p}_{20},\bm{d}_{2}\}$の共通垂線の両端点$(\bm{p}_{\mathrm{P}1}, \bm{p}_{\mathrm{P}2})(\mathrm{L}_{1},\mathrm{L}_{2})$}(3Dのみ)

$(\bm{p}_{\mathrm{P}1}, \bm{p}_{\mathrm{P}2})(\mathrm{L}_{1},\mathrm{L}_{2})=\begin{cases}
(\bm{p}(s;\mathrm{L}_{1}),\bm{p}(t;\mathrm{L}_{2})) & (D\geq\varepsilon\mbox{のとき}) \\
(\bm{p}_{10},\bm{p}_{\mathrm{P}}(\bm{p}_{10};\mathrm{L}_{2})) & (D<\varepsilon\mbox{のとき})
\end{cases}$.
ただし,
\begin{align*}
r_{11}&=\bm{d}_{1}^{\mathrm{T}}\bm{d}_{1} \\
r_{12}&=\bm{d}_{1}^{\mathrm{T}}\bm{d}_{2} \\
r_{22}&=\bm{d}_{2}^{\mathrm{T}}\bm{d}_{2} \\
D&=r_{11}r_{22}-r_{12}^{2} \\
s&=(r_{12}\bm{d}_{2}-r_{22}\bm{d}_{1})^{\mathrm{T}}(\bm{p}_{20}-\bm{p}_{10})/D \\
t&=(r_{11}\bm{d}_{2}-r_{12}\bm{d}_{1})^{\mathrm{T}}(\bm{p}_{20}-\bm{p}_{10})/D
\end{align*}

\noindent{\tt zLine3DCommonPerp}
{\bf 2直線$\mathrm{L}_{1}\{\bm{p}_{10},\bm{d}_{1}\}$,$\mathrm{L}_{2}\{\bm{p}_{20},\bm{d}_{2}\}$の共通垂線}(3Dのみ)

$\mathrm{L}\left<\bm{p}_{\mathrm{P}1},\bm{p}_{\mathrm{P}2}\right>$

\noindent{\tt zDistLine3DLine3D}
{\bf 2直線$\mathrm{L}_{1}\{\bm{p}_{10},\bm{d}_{1}\}$,$\mathrm{L}_{2}\{\bm{p}_{20},\bm{d}_{2}\}$間の距離$d(\mathrm{L}_{1},\mathrm{L}_{2})$}(3Dのみ)

$d(\mathrm{L}_{1},\mathrm{L}_{2})=\|\bm{p}_{\mathrm{P}1}-\bm{p}_{\mathrm{P}2}\|$


\section{線分}

線分$\mathrm{E}$は,その両端点の位置ベクトル$\bm{p}_{1}$,$\bm{p}_{2}$で表します($\mathrm{E}\{\bm{p}_{1},\bm{p}_{2}\}$).
内部変数として結合ベクトル$\bm{v}\overset{\mathrm{def}}{=}\bm{p}_{2}-\bm{p}_{1}$を保持します.
次のメソッドが用意されています.

\noindent{\tt zEdge2D/3DLen}
{\bf $\mathrm{E}$の長さ$l(\mathrm{E})$}

$l(\mathrm{E})=\|\bm{v}\|$

\noindent{\tt zEdge2D/3DDir}
{\bf $\mathrm{E}$の単位方向ベクトル$\bm{d}(\mathrm{E})$}

$\bm{d}(\mathrm{E})=\bm{v}/l(\mathrm{E})$

\noindent{\tt zEdge2D/3DToLine}
{\bf $\mathrm{E}$の同一直線$\mathrm{L}\left\{\mathrm{E}\right\}$}

$\mathrm{L}\left\{\mathrm{E}\right\}=\mathrm{L}\left<\bm{p}_{1},\bm{p}_{2}\right>$

\noindent{\tt zEdge2D/3DProjPoint}
{\bf ある点$\bm{p}$の$\mathrm{E}$の同一直線への射影$\bm{p}_{\mathrm{P}}(\bm{p};\mathrm{E})$}

$\bm{p}_{\mathrm{P}}(\bm{p};\mathrm{E})=\bm{p}_{\mathrm{P}}(\bm{p};\mathrm{L}\left\{\mathrm{E}\right\})$

\noindent{\tt zEdge2D/3DClosest}
{\bf ある点$\bm{p}$の$\mathrm{E}$上の近接点$\bm{p}_{\mathrm{C}}(\bm{p};\mathrm{E})$}

$\bm{p}_{\mathrm{C}}(\bm{p};\mathrm{E})=\begin{cases}
\bm{p}_{1} & (s<0) \\
\bm{p}_{\mathrm{P}}(\bm{p};\mathrm{E}) & (0\leq s\leq l(\mathrm{E})) \\
\bm{p}_{2} & (s>l(\mathrm{E}))
\end{cases}$
(ただし,$s=\bm{d}(\mathrm{E})^{\mathrm{T}}\varDelta\bm{p}$,
$\varDelta\bm{p}\overset{\mathrm{def}}{=}\bm{p}-\bm{p}_{1}$)

\noindent{\tt zEdge2D/3DPointIsOn}
{\bf ある点$\bm{p}$が$\mathrm{E}$上に載っているか否か(マージン$d_{\mathrm{m}}$)}

$\bm{p}_{\mathrm{C}}(\bm{p};\mathrm{E})<d_{\mathrm{m}}$

\noindent{\tt zEdge2D/3DContigVert}
{\bf ある点$\bm{p}$に近い端点$\bm{p}_{\mathrm{CV}}(\bm{p};\mathrm{E})$}

$\bm{p}_{\mathrm{CV}}(\bm{p};\mathrm{E})=\begin{cases}
\bm{p}_{1} & (\|\bm{p}-\bm{p}_{1}\|<\|\bm{p}-\bm{p}_{2}\|\mbox{のとき}) \\
\bm{p}_{2} & (\|\bm{p}-\bm{p}_{1}\|\geq\|\bm{p}-\bm{p}_{2}\|\mbox{のとき})
\end{cases}$

\noindent{\tt zEdge2D/3DDistFromPoint}
{\bf ある点$\bm{p}$と$\mathrm{E}$との距離$l(\bm{p};\mathrm{E})$}

$l(\bm{p};\mathrm{E})=\|\bm{p}-\bm{p}_{\mathrm{C}}(\bm{p};\mathrm{E})\|$



\section{平面}

平面$P$は3次元空間のみに定義される構造で,その平面が通るある一点の位置ベクトル$\bm{p}_{0}$と単位法線ベクトル$\bm{n}$で表します($P\{\bm{p}_{0},\bm{\nu}\}$).
次のメソッドが用意されています.

\noindent{\tt zPlane3DDistFromPoint}
{\bf ある点$\bm{p}$と$\mathrm{P}$との符号付き距離$d(\bm{p};\mathrm{P})$}

$d(\bm{p};\mathrm{P})=(\bm{p}-\bm{p}_{0})^{\mathrm{T}}\bm{n}$

\noindent{\tt zPlane3DPointIsOn}
{\bf ある点$\bm{p}$が$\mathrm{P}$上に載っているか否か(マージン$d_{\mathrm{m}}$)}

$\left|d(\bm{p};\mathrm{P})\right|<d_{\mathrm{m}}$

\noindent{\tt zPlane3DProjPoint}
{\bf ある点$\bm{p}$の$\mathrm{P}$への射影$\bm{p}_{\mathrm{P}}(\bm{p};\mathrm{P})$}

$\bm{p}_{\mathrm{P}}(\bm{p};\mathrm{P})=\bm{p}-d(\bm{p};\mathrm{P})\bm{n}$



\section{三角形}

三角形$T$は,その3頂点の位置ベクトル$\bm{p}_{1}$,$\bm{p}_{2}$,$\bm{p}_{3}$で表します($T\{\bm{p}_{1},\bm{p}_{2},\bm{p}_{3}\}$).
次のメソッドが用意されています.

\noindent{\tt zTri3DArea}
{\bf $\mathrm{T}$の面積$s(\mathrm{T})$}

$s(\mathrm{T})=\left\|(\bm{p}_{2}-\bm{p}_{1})\times(\bm{p}_{3}-\bm{p}_{1})\right\|$

\noindent{\tt zTri3DNorm}
{\bf $\mathrm{T}$の単位法線ベクトル$\bm{n}(\mathrm{T})$}(3Dのみ)

$\bm{n}(\mathrm{T})=\frac{(\bm{p}_{2}-\bm{p}_{1})\times(\bm{p}_{3}-\bm{p}_{1})}{s(\mathrm{T})}$

\noindent{\tt zTri3DBarycenter}
{\bf $\mathrm{T}$の重心$\bm{p}_{\mathrm{GC}}(\mathrm{T})$}

$\bm{p}_{\mathrm{GC}}(\mathrm{T})=\frac{\displaystyle \bm{p}_{1}+\bm{p}_{2}+\bm{p}_{3}}{3}$

\noindent{\tt zTri3DCircumcenter}
{\bf $\mathrm{T}$の外心$\bm{p}_{\mathrm{CC}}(\mathrm{T})$}

$\bm{p}_{\mathrm{GC}}(\mathrm{T})=\frac{\displaystyle s_{1}\bm{p}_{1}+s_{2}\bm{p}_{2}+s_{3}\bm{p}_{3}}{s_{1}+s_{2}+s_{3}}$

ただし,
\begin{align*}
s_{1}&=r_{1}(r_{2}+r_{3}-r_{1})
\\
s_{2}&=r_{2}(r_{3}+r_{1}-r_{2})
\\
s_{3}&=r_{3}(r_{1}+r_{2}-r_{3})
\\
r_{1}&=\|\bm{p}_{3}-\bm{p}_{2}\|^{2}
\\
r_{2}&=\|\bm{p}_{1}-\bm{p}_{3}\|^{2}
\\
r_{3}&=\|\bm{p}_{2}-\bm{p}_{1}\|^{2}
\end{align*}

\noindent{\tt zTri3DIncenter}
{\bf $\mathrm{T}$の内心$\bm{p}_{\mathrm{IC}}(\mathrm{T})$}

$\bm{p}_{\mathrm{IC}}(\mathrm{T})=\frac{\displaystyle w_{1}\bm{p}_{1}+w_{2}\bm{p}_{2}+w_{3}\bm{p}_{3}}{w_{1}+w_{2}+w_{3}}$

ただし,
\begin{align*}
w_{1}&=\|\bm{p}_{2}-\bm{p}_{3}\|
\\
w_{2}&=\|\bm{p}_{3}-\bm{p}_{1}\|
\\
w_{3}&=\|\bm{p}_{1}-\bm{p}_{2}\|
\end{align*}

\noindent{\tt zTri3DOrthocenter}
{\bf $\mathrm{T}$の垂心$\bm{p}_{\mathrm{OC}}(\mathrm{T})$}

$\bm{p}_{\mathrm{OC}}(\mathrm{T})=\bm{p}_{\mathrm{CC}}(\mathrm{T}^{\prime})$

ただし,
\begin{align*}
\mathrm{T}^{\prime}&=\mathrm{T}^{\prime}\{\bm{p}_{1}^{\prime},\bm{p}_{2}^{\prime},\bm{p}_{3}^{\prime}\}
\\
\bm{p}_{1}^{\prime}&=\bm{p}_{2}+\bm{p}_{3}-\bm{p}_{1}
\\
\bm{p}_{2}^{\prime}&=\bm{p}_{3}+\bm{p}_{1}-\bm{p}_{2}
\\
\bm{p}_{3}^{\prime}&=\bm{p}_{1}+\bm{p}_{2}-\bm{p}_{3}
\end{align*}

\noindent{\tt zTri2D/3DContigVert}
{\bf ある点$\bm{p}$に近い頂点$\bm{p}_{\mathrm{CV}}(\bm{p};\mathrm{T})$}

$\bm{p}_{\mathrm{CV}}(\bm{p};\mathrm{T}\{\bm{p}_{1},\bm{p}_{2},\bm{p}_{3}\})=\mathop{\mathrm{arg~min}}_{\bm{p}_{i}}\left\{\|\bm{p}-\bm{p}_{i}\|\,|\,i=1,2,3 \right\}$

\noindent{\tt zTri3DToPlane}
{\bf $\mathrm{T}$の同一平面$\mathrm{P}\left\{\mathrm{T}\right\}$}(3Dのみ)

$\mathrm{P}\left\{\mathrm{T}\right\}=\mathrm{P}\{\bm{p}_{1},\bm{n}(\mathrm{T})\}$

\noindent{\tt zTri3DDistFromPointToPlane}
{\bf ある点$\bm{p}$と$\mathrm{T}$の同一平面との符号付き距離$d_{\mathrm{S}}(\bm{p};\mathrm{T})$}

$d_{\mathrm{S}}(\bm{p};\mathrm{T})=d(\bm{p};\mathrm{P}\{\mathrm{T}\})$

\noindent{\tt zTri3DPointIsOnPlane}
{\bf ある点$\bm{p}$が$\mathrm{T}$の同一平面上に載っているか否か(マージン$d_{\mathrm{m}}$)}

$|d_{\mathrm{S}}(\bm{p};\mathrm{T})|<d_{\mathrm{m}}$

\noindent{\tt zTri3DProjPoint}
{\bf ある点$\bm{p}$の$\mathrm{T}$の同一平面への射影$\bm{p}_{\mathrm{P}}(\bm{p};\mathrm{T})$}

$\bm{p}_{\mathrm{P}}(\bm{p};\mathrm{T})=\bm{p}_{\mathrm{P}}(\bm{p};\mathrm{P}\{\mathrm{T}\})$

\noindent{\tt zTri3DPointIsInside}
{\bf ある点$\bm{p}$の$\mathrm{T}$の同一平面への射影が$\mathrm{T}$の内部にあるか否か(マージン$d_{\mathrm{m}}$)}

$(\bm{p}-\bm{p}_{1})^{\top}\bm{g}_{1}<d_{\mathrm{m}}\wedge
(\bm{p}-\bm{p}_{2})^{\top}\bm{g}_{2}<d_{\mathrm{m}}\wedge
(\bm{p}-\bm{p}_{3})^{\top}\bm{g}_{3}<d_{\mathrm{m}}$

ただし,
\begin{align*}
\bm{g}_{1}&=\frac{(\bm{p}_{2}-\bm{p}_{1})\times\bm{n}(\mathrm{T})}{\|(\bm{p}_{2}-\bm{p}_{1})\times\bm{n}(\mathrm{T})\|}
\\
\bm{g}_{2}&=\frac{(\bm{p}_{3}-\bm{p}_{2})\times\bm{n}(\mathrm{T})}{\|(\bm{p}_{3}-\bm{p}_{2})\times\bm{n}(\mathrm{T})\|}
\\
\bm{g}_{3}&=\frac{(\bm{p}_{1}-\bm{p}_{3})\times\bm{n}(\mathrm{T})}{\|(\bm{p}_{1}-\bm{p}_{3})\times\bm{n}(\mathrm{T})\|}
\end{align*}

\noindent{\tt zTri3DClosest}
{\bf ある点$\bm{p}$の$\mathrm{T}$上の近接点$\bm{p}_{\mathrm{C}}(\bm{p};\mathrm{T})$}

$\bm{p}_{\mathrm{C}}(\bm{p};\mathrm{T})=\begin{cases}
\bm{p}_{\mathrm{P}}(\bm{p};\mathrm{T}) & (\bm{p}_{\mathrm{P}}(\bm{p};\mathrm{T})\in\mathrm{T}) \\
\bm{p}_{\mathrm{C}}(\bm{p};\mathrm{E}_{i^{*}}) & (\mbox{otherwise})
\end{cases}$

ただし,
\begin{align*}
i^{*}&=\mathop{\mathrm{arg~min}}_{i}\left\{l(\bm{p};\mathrm{E}_{i})\,|\,i=1,2,3\right\}
\\
\mathrm{E}_{1}&=\mathrm{E}_{1}\{\bm{p}_{1},\bm{p}_{2}\}
\\
\mathrm{E}_{2}&=\mathrm{E}_{2}\{\bm{p}_{2},\bm{p}_{3}\}
\\
\mathrm{E}_{3}&=\mathrm{E}_{3}\{\bm{p}_{3},\bm{p}_{1}\}
\end{align*}

\noindent{\tt zTri3DClosest}
{\bf ある点$\bm{p}$と$\mathrm{T}$の距離$d(\bm{p};\mathrm{T})$}

$d(\bm{p};\mathrm{T})=\|\bm{p}-\bm{p}_{\mathrm{C}}(\bm{p};\mathrm{T})\|$


\end{document}
