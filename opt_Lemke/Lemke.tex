\documentclass[a4paper]{jsarticle}

\usepackage{amsmath,amssymb,bm}
\usepackage{graphicx}
\usepackage{ascmac}
\usepackage{algorithm,algpseudocode}
\everymath{\displaystyle}

\title{相補掃き出し法(Lemke法) --- 線形相補性問題と凸2次計画問題の解法}

\begin{document}
\maketitle

\section{線形相補性問題とは}

{\bf 線形相補性問題(Linear Complementary Problem, LCP)}とは,次の条件を満たす変数$\boldsymbol{z}$を求める問題です.
\begin{align*}
\boldsymbol{0}\leq\boldsymbol{z}\perp\boldsymbol{M}\boldsymbol{z}+\boldsymbol{p}\geq\boldsymbol{0}
\end{align*}
ただし,
$\boldsymbol{M}$は非負定値対称行列,$\boldsymbol{p}$は定数ベクトルです.
あるいは次のようにも書けます.
\begin{align*}
\boldsymbol{w}=\boldsymbol{M}\boldsymbol{z}+\boldsymbol{p},\quad
\boldsymbol{w}\geq\boldsymbol{0},\quad
\boldsymbol{z}\geq\boldsymbol{0},\quad
\boldsymbol{w}^{\mathrm{T}}\boldsymbol{z}=0
\end{align*}
$\boldsymbol{w}=[w_{1}~\cdots~w_{n}]^{\mathrm{T}}$と$\boldsymbol{z}=[z_{1}~\cdots~z_{n}]^{\mathrm{T}}$
($n$は$\boldsymbol{z}$の次元数)は同数の成分を持ちますが,
$\boldsymbol{w}\geq\boldsymbol{0}$かつ$\boldsymbol{z}\geq\boldsymbol{0}$なので,
$\boldsymbol{w}^{\mathrm{T}}\boldsymbol{z}=0$となるためには
$w_{i}$と$z_{i}$($\forall i\in\{1,\cdots,n\}$)のどちらか一方は必ず0になる,
ということが「相補性」の意味です.
ちょっと変わったクラスの問題に見えるかも知れませんが,
後ほど記すように,
ポピュラーな凸2次計画問題も双対性に基づいて線形相補性問題に等価変換できるので,
わりと様々な事例がこのように表せます.


{\bf 相補掃き出し法}({\bf Complementary Pivotting Method}, {\bf Lemke法})は,
線形相補性問題の標準的解法なのですが,
その説明がウェブ上になかなか見当たらないので自分で書くことにしました.
なお,本記事の作成にあたって次の書籍を参考にしました.

\begin{screen}
岩波講座応用数学[方法7] 最適化法,藤田 宏,今野 浩,田邉 國士著,岩波書店,1994
\end{screen}


\section{相補掃き出し法のアルゴリズム}

まず,次の集合$\mathcal{Q}$を定義します.
\begin{align*}
\mathcal{Q}=\left\{(\boldsymbol{w},\boldsymbol{z})\left|
\boldsymbol{w}-\boldsymbol{M}\boldsymbol{z}=\boldsymbol{p},\boldsymbol{w}\geq\boldsymbol{0}, \boldsymbol{z}\geq\boldsymbol{0}
\right.\right\}
\end{align*}
$\mathcal{Q}$の元のうち
条件$\boldsymbol{w}^{\mathrm{T}}\boldsymbol{z}=0$を満たすものが線形相補性問題の解です.
もし$\boldsymbol{p}\geq\boldsymbol{0}$ならば,
$\boldsymbol{w}=\boldsymbol{p}$,$\boldsymbol{z}=\boldsymbol{0}$は自明解となることに注意して下さい.
以降では$\boldsymbol{p}\ngeq\boldsymbol{0}$を仮定します.

天下り的ですが,ここで人為変数$v$を新たに導入して,
上記の$\mathcal{Q}$と似た次の集合$\tilde{\mathcal{Q}}$を定義します.
\begin{align*}
\tilde{\mathcal{Q}}=\left\{(\boldsymbol{w},\boldsymbol{z},v)\left|
\boldsymbol{w}-\boldsymbol{M}\boldsymbol{z}-v\boldsymbol{l}=\boldsymbol{p}, \boldsymbol{w}\geq\boldsymbol{0}, \boldsymbol{z}\geq\boldsymbol{0}, v\geq 0
\right.\right\}
\end{align*}
ただし,
$\boldsymbol{l}\overset{\mathrm{def}}{=}[1~\cdots~1]^{\mathrm{T}}$とおきました.
$(\boldsymbol{w},\boldsymbol{z},0)\in\tilde{\mathcal{Q}}$であれば$(\boldsymbol{w},\boldsymbol{z})\in\mathcal{Q}$であることは,
すぐにご理解頂けるともの思います.
ここで,$\boldsymbol{p}=[p_{1}~\cdots~p_{n}]^{\mathrm{T}}$に対し
$s=\mathop{\mathrm{arg~min}}_{i}\left\{p_{i}\right\}$とおけば,
$(\boldsymbol{w},\boldsymbol{z},v)=(\boldsymbol{p}-p_{s}\boldsymbol{l},\boldsymbol{0},-p_{s})$は$\tilde{\mathcal{Q}}$の元となり
($\boldsymbol{p}\ngeq\boldsymbol{0}$を仮定しているので$-p_{s}>0$です),
かつ$\boldsymbol{w}^{\mathrm{T}}\boldsymbol{z}=0$を満たします.
この操作によって,今度は$w_{s}$が$0$になりますので,
それと相補関係にある$z_{s}$を$0$から増やしても$\boldsymbol{w}^{\mathrm{T}}\boldsymbol{z}=0$は維持されることになります.

上記のことを,次のような表(タブロー)を作ってもう一度考えてみます.
\begin{align*}
\left|\begin{array}{c|c|c|c|c|c|c|c|c|c|c|c|c|c|c}
1      & 2      & \cdots & s-1    & s      & s+1    & \cdots & n      & n+1       & \cdots & n+s       & \cdots & 2n        & 2n+1   & \\
\hline
1      & 0      &        & 0      & 0      & 0      &        & 0      & -m_{11}    &        & -m_{1s}    & \cdots & -m_{1n}    & -1     & p_{1} \\
0      & 1      &        & \vdots & \vdots & \vdots &        &        & \vdots    &        & \vdots    &        & \vdots    & \vdots & \vdots \\
       & 0      &        & 0      &        &        &        &        &           &        &           &        &           &        &        \\
       &        &        & 1      & 0      &        &        &        & -m_{(s-1)1} &        & -m_{(s-1)s} &        & -m_{(s-1)n} & -1     & p_{s-1} \\
\vdots & \vdots & \cdots & 0      & 1      & 0      & \cdots & \vdots & -m_{s1}    & \cdots & -m_{ss}    & \cdots & -m_{sn}    & -1     & p_{s} \\
       &        &        &        & 0      & 1      &        &        & -m_{(s+1)1} &        & -m_{(s+1)s} &        & -m_{(s+1)n} & -1     & p_{s+1} \\
       &        &        & \vdots &        & 0      &        &        & \vdots    &        & \vdots    &        & \vdots    & \vdots & \vdots \\
       &        &        &        & \vdots & \vdots &        & 0      &           &        &           &        &           &        &        \\
0      & 0      &        & 0      & 0      & 0      &        & 1      & -m_{n1}    &        & -m_{ns}    &        & -m_{nn}    & -1     & p_{n} \\
\end{array}\right|
\end{align*}
ただし,
$m_{ij}$は$\boldsymbol{M}$の$i$行目$j$列目成分です.
$1\sim n$列目が$\boldsymbol{w}$に,
$n+1\sim 2n$列目が$\boldsymbol{z}$に,
$2n+1$列目が$v$にそれぞれ対応します.
最右列は$\boldsymbol{p}$です.
$s=\mathop{\mathrm{arg~min}}_{i}\left\{p_{i}\right\}$を求め,
$s$行$2n+1$列目を軸として掃き出しを行うと,次のようになります.
\begin{align*}
\left|\begin{array}{c|c|c|c|c|c|c|c|c|c|c|c|c|c|c}
1      & 2      & \cdots & s-1    & s      & s+1    & \cdots & n      & n+1            & \cdots & n+s            & \cdots & 2n             & 2n+1   & \\
\hline
1      & 0      &        & 0      &-1      & 0      &        & 0      & m_{s1}-m_{11}    &        & m_{ss}-m_{1s}    & \cdots & m_{sn}-m_{1n}    & 0      & p_{1}-p_{s} \\
0      & 1      &        & \vdots & \vdots & \vdots &        &        & \vdots         &        & \vdots         &        & \vdots         & \vdots & \vdots \\
       & 0      &        & 0      &        &        &        &        &                &        &                &        &                &        &        \\
       &        &        & 1      &-1      &        &        &        & m_{s1}-m_{(s-1)1} &        & m_{ss}-m_{(s-1)s} &        & m_{sn}-m_{(s-1)n} & 0      & p_{s-1}-p_{s} \\
\vdots & \vdots & \cdots & 0      &-1      & 0      & \cdots & \vdots & m_{s1}          & \cdots & m_{ss}          & \cdots & m_{sn}          & 1      &-p_{s} \\
       &        &        &        &-1      & 1      &        &        & m_{s1}-m_{(s+1)1} &        & m_{ss}-m_{(s+1)s} &        & m_{sn}-m_{(s+1)n} & 0      & p_{s+1}-p_{s} \\
       &        &        & \vdots &        & 0      &        &        & \vdots         &        & \vdots         &        & \vdots         & \vdots & \vdots \\
       &        &        &        & \vdots & \vdots &        & 0      &                &        &                &        &                &        &        \\
0      & 0      &        & 0      &-1      & 0      &        & 1      & m_{s1}-m_{n1}    &        & m_{ss}-m_{ns}    &        & m_{sn}-m_{nn}    & 0      & p_{n}-p_{s} \\
\end{array}\right|
\end{align*}
そして,
$2n+1$列目を$s$列目に,
$s$列目を$n+s$列目に,
$n+s$列目を$2n+1$列目にそれぞれ移動します.
\begin{align*}
\left|\begin{array}{c|c|c|c|c|c|c|c|c|c|c|c|c|c|c}
1      & 2      & \cdots & s-1    & 2n+1   & s+1    & \cdots & n      & n+1            & \cdots & s      & \cdots & 2n             & n+s            & \\
\hline
1      & 0      &        & 0      & 0      & 0      &        & 0      & m_{s1}-m_{11}    &        & -1     & \cdots & m_{sn}-m_{1n}    & m_{ss}-m_{1s}    & p_{1}-p_{s} \\
0      & 1      &        & \vdots & \vdots & \vdots &        &        & \vdots         &        & \vdots &        & \vdots         & \vdots         & \vdots \\
       & 0      &        & 0      &        &        &        &        &                &        &        &        &                &                &        \\
       &        &        & 1      & 0      &        &        &        & m_{s1}-m_{(s-1)1} &        & -1     &        & m_{sn}-m_{(s-1)n} & m_{ss}-m_{(s-1)s} & p_{s-1}-p_{s} \\
\vdots & \vdots & \cdots & 0      & 1      & 0      & \cdots & \vdots & m_{s1}          & \cdots & -1     & \cdots & m_{sn}          & m_{ss}          &-p_{s} \\
       &        &        &        & 0      & 1      &        &        & m_{s1}-m_{(s+1)1} &        & -1     &        & m_{sn}-m_{(s+1)n} & m_{ss}-m_{(s+1)s} & p_{s+1}-p_{s} \\
       &        &        & \vdots &        & 0      &        &        & \vdots         &        & \vdots &        & \vdots         & \vdots         & \vdots \\
       &        &        &        & \vdots & \vdots &        & 0      &                &        &        &        &                &                &        \\
0      & 0      &        & 0      & 0      & 0      &        & 1      & m_{s1}-m_{n1}    &        & -1     &        & m_{sn}-m_{nn}    & m_{ss}-m_{ns}    & p_{n}-p_{s} \\
\end{array}\right|
\end{align*}
これで,列の入れ替えは起こったものの形式的には最初と同じ形になりました.
左から$n$列目までの部分は単位行列となっており,最右列の成分が全て$0$以上なので,
\begin{itemize}
\item $w_{i}=p_{i}-p_{s}$($i=1,\cdots,s-1,s+1,\cdots,n$),$v=-p_{s}$,$w_{s}=z_{i}=0$($i=1,\cdots,n$,$i\neq s$)が$\tilde{\mathcal{Q}}$の元
\item $w_{s}=0$なので,$z_{s}$は非負となる範囲で自由に値を変えられる
\end{itemize}
というのがこの表の読み方です.

このことを一般化して考えてみましょう.
$\boldsymbol{w}$,$\boldsymbol{z}$,$v$をまとめて
$\boldsymbol{q}=[\boldsymbol{w}^{\mathrm{T}}~\boldsymbol{z}^{\mathrm{T}}~v]^{\mathrm{T}}=[q_{1}~\cdots~q_{2n+1}]^{\mathrm{T}}$とおきます.
これらが,次の式を満たす変数群$\boldsymbol{q}_{\mathrm{B}}$,$\boldsymbol{q}_{\mathrm{N}}$,$q_{\mathrm{D}}$に分けられる状況を考えます.
\begin{align*}
\boldsymbol{q}_{\mathrm{B}}+\boldsymbol{H}\boldsymbol{q}_{\mathrm{N}}+q_{\mathrm{D}}\boldsymbol{d}=\boldsymbol{r}
\end{align*}
ただし,
$\boldsymbol{H}$は定数行列,
$\boldsymbol{d}$は定数ベクトル,
$\boldsymbol{r}$は全ての成分が非負の定数ベクトルで,
$\boldsymbol{q}_{\mathrm{B}}$と$\boldsymbol{q}_{\mathrm{N}}$はともに$n$次ベクトルであるものとします.
$\boldsymbol{q}_{\mathrm{B}}$を{\bf 基底変数},
$\boldsymbol{q}_{\mathrm{N}}$を{\bf 非基底変数},
$q_{\mathrm{D}}$を{\bf 駆動変数}とそれぞれ呼びます.
$\boldsymbol{q}_{\mathrm{B}}$と$\boldsymbol{q}_{\mathrm{N}}$,$q_{\mathrm{D}}$のインデックスの集合を
$\mathcal{I}_{\mathrm{B}}=\left\{i_{\mathrm{B}1},\cdots,i_{\mathrm{B}n}\right\}$,
$\mathcal{I}_{\mathrm{N}}=\left\{i_{\mathrm{N}1},\cdots,i_{\mathrm{N}n}\right\}$,
$\mathcal{I}_{\mathrm{D}}=\left\{i_{\mathrm{D}}\right\}$とそれぞれおき,
\begin{align*}
\boldsymbol{q}_{\mathrm{B}}=\begin{bmatrix}
q_{i_{\mathrm{B}1}} \\
 \vdots       \\
q_{i_{\mathrm{B}n}} \\
\end{bmatrix}
,\quad
\boldsymbol{q}_{\mathrm{N}}=\begin{bmatrix}
q_{i_{\mathrm{N}1}} \\
 \vdots       \\
q_{i_{\mathrm{N}n}} \\
\end{bmatrix}
,\quad
\boldsymbol{H}=\begin{bmatrix}
h_{1} & \cdots & h_{1n} \\
 \vdots        & \ddots & \vdots \\
h_{n1} & \cdots & h_{nn} \\
\end{bmatrix}
,\quad
\boldsymbol{d}=\begin{bmatrix}
d_{1} \\
 \vdots       \\
d_{n} \\
\end{bmatrix}
,\quad
\boldsymbol{r}=\begin{bmatrix}
r_{1} \\
 \vdots       \\
r_{n} \\
\end{bmatrix}
\end{align*}
と成分表示すると,タブローは次のように書けます.
\begin{align*}
\left|\begin{array}{c|c|c|c|c|c|c|c|c|c|c|c|c|c|c}
i_{\mathrm{B}1} & i_{\mathrm{B}2} & \cdots & i_{\mathrm{B}(s-1)} & i_{\mathrm{B}s} & i_{\mathrm{B}(s+1)} & \cdots & i_{\mathrm{B}n} & i_{\mathrm{N}1} & \cdots & i_{\mathrm{N}s} & \cdots & i_{\mathrm{N}n} & i_{\mathrm{D}} & \\
\hline
1      & 0      &        & 0      & 0      & 0      &        & 0      & h_{11}    &        & h_{1s}    & \cdots & h_{1n}    & d_{1}   & r_{1} \\
0      & 1      &        & \vdots & \vdots & \vdots &        &        & \vdots   &        & \vdots   &        & \vdots   & \vdots & \vdots \\
       & 0      &        & 0      &        &        &        &        &          &        &          &        &          &        &        \\
       &        &        & 1      & 0      &        &        &        & h_{(s-1)1} &        & h_{(s-1)s} &        & h_{(s-1)n} & d_{s-1}  & r_{s-1} \\
\vdots & \vdots & \cdots & 0      & 1      & 0      & \cdots & \vdots & h_{s1}    & \cdots & h_{ss}    & \cdots & h_{sn}    & d_{s}   & r_{s} \\
       &        &        &        & 0      & 1      &        &        & h_{(s+1)1} &        & h_{(s+1)s} &        & h_{(s+1)n} & d_{s+1}  & r_{s+1} \\
       &        &        & \vdots &        & 0      &        &        & \vdots   &        & \vdots   &        & \vdots   & \vdots & \vdots \\
       &        &        &        & \vdots & \vdots &        & 0      &          &        &          &        &          &        &        \\
0      & 0      &        & 0      & 0      & 0      &        & 1      & h_{n1}    &        & h_{ns}    &        & h_{nn}    & d_{n}   & r_{n} \\
\end{array}\right|
\end{align*}
軸をどのように選ぶかはいったん置いておくとして,
$s$行目$i_{\mathrm{D}}$列目に選んで掃き出しを行うと,次のようになります.
\begin{align*}
\left|\begin{array}{c|c|c|c|c|c|c|c|c|c|c|c|c|c|c}
i_{\mathrm{B}1} & i_{\mathrm{B}2} & \cdots & i_{\mathrm{B}(s-1)} & i_{\mathrm{B}s} & i_{\mathrm{B}(s+1)} & \cdots & i_{\mathrm{B}n} & i_{\mathrm{N}1} & \cdots & i_{\mathrm{N}s} & \cdots & i_{\mathrm{N}n} & i_{\mathrm{D}} & \\
\hline
1      & 0      &        & 0      &-d_{1}/d_{s}  & 0      &        & 0      & h_{11}^{\prime}    &        & h_{1s}^{\prime}    & \cdots & h_{1n}^{\prime}    & 0      & r_{1}^{\prime} \\
0      & 1      &        & \vdots & \vdots      & \vdots &        &        & \vdots          &        & \vdots          &        & \vdots          & \vdots & \vdots \\
       & 0      &        & 0      &             &        &        &        &                 &        &                 &        &                 &        &        \\
       &        &        & 1      &-d_{s-1}/d_{s} &        &        &        & h_{(s-1)1}^{\prime} &        & h_{(s-1)s}^{\prime} &        & h_{(s-1)n}^{\prime} & 0      & r_{s-1}^{\prime} \\
\vdots & \vdots & \cdots & 0      & 1/d_{s}      & 0      & \cdots & \vdots & h_{s1}^{\prime}    & \cdots & h_{ss}^{\prime}    & \cdots & h_{sn}^{\prime}    & 1      & r_{s}^{\prime} \\
       &        &        &        &-d_{s+1}/d_{s} & 1      &        &        & h_{(s+1)1}^{\prime} &        & h_{(s+1)s}^{\prime} &        & h_{(s+1)n}^{\prime} & 0      & r_{s+1}^{\prime} \\
       &        &        & \vdots &             & 0      &        &        & \vdots          &        & \vdots          &        & \vdots          & \vdots & \vdots \\
       &        &        &        & \vdots      & \vdots &        & 0      &                 &        &                 &        &                 &        &        \\
0      & 0      &        & 0      &-d_{n}/d_{s}  & 0      &        & 1      & h_{n1}^{\prime}    &        & h_{ns}^{\prime}    &        & h_{nn}^{\prime}    & 0      & r_{n}^{\prime} \\
\end{array}\right|
\end{align*}
ただし,
\begin{align*}
h_{ij}^{\prime}&=h_{ij}-(d_{i}/d_{s})h_{sj} \quad (i=1,\cdots,n, i\neq s) \\
h_{sj}^{\prime}&=h_{sj}/d_{s} \\
r_{i}^{\prime}&=r_{i}-(d_{i}/d_{s})r_{s} \\
r_{s}^{\prime}&=r_{s}/d_{s}
\end{align*}
とおきました.
このように変形しても$r_{i}^{\prime}\geq 0$($i=1,\cdots,n$)が満たされるためには,
$s=\mathop{\mathrm{arg~min}}_{i}\left\{\left.r_{i}/d_{i}\right|d_{i}>0 \right\}$でなければなりません.
もし全ての$i=1,\cdots,n$に対して$d_{i}\leq 0$ならば,元の相補性問題に解は無いということになので,計算を停止します.

軸となる$s$が見つかり上記の掃き出しを行ったら,
$i_{\mathrm{D}}$列目を$i_{\mathrm{B}s}$列目に,
$i_{\mathrm{B}s}$列目を$i_{\mathrm{N}s}$列目に,
$i_{\mathrm{N}s}$列目を$i_{\mathrm{D}}$列目にそれぞれ移動します.
\begin{align*}
\left|\begin{array}{c|c|c|c|c|c|c|c|c|c|c|c|c|c|c}
i_{\mathrm{B}1} & i_{\mathrm{B}2} & \cdots & i_{\mathrm{D}} & i_{\mathrm{B}s} & i_{\mathrm{B}(s+1)} & \cdots & i_{\mathrm{B}n} & i_{\mathrm{N}1} & \cdots & i_{\mathrm{B}s} & \cdots & i_{\mathrm{N}n} & i_{\mathrm{N}s} & \\
\hline
1      & 0      &        & 0      & 0           & 0      &        & 0      & h_{11}^{\prime}    &        &-d_{1}/d_{s}    & \cdots & h_{1n}^{\prime} & h_{1s}^{\prime}  & r_{1}^{\prime} \\
0      & 1      &        & \vdots & \vdots      & \vdots &        &        & \vdots          &        & \vdots          &        & \vdots          & \vdots & \vdots \\
       & 0      &        & 0      &             &        &        &        &                 &        &                 &        &                 &        &        \\
       &        &        & 1      & 0           &        &        &        & h_{(s-1)1}^{\prime} &        &-d_{s-1}/d_{s}     &        & h_{(s-1)n}^{\prime} & h_{(s-1)s}^{\prime} & r_{s-1}^{\prime} \\
\vdots & \vdots & \cdots & 0      & 1      & 0      & \cdots & \vdots & h_{s1}^{\prime}    & \cdots & 1/d_{s}   & \cdots & h_{sn}^{\prime}    & h_{ss}^{\prime}     & r_{s}^{\prime} \\
       &        &        &        & 0 & 1      &        &        & h_{(s+1)1}^{\prime} &        &-d_{s+1}/d_{s} &        & h_{(s+1)n}^{\prime} & h_{(s+1)s}^{\prime}     & r_{s+1}^{\prime} \\
       &        &        & \vdots &             & 0      &        &        & \vdots          &        & \vdots          &        & \vdots          & \vdots & \vdots \\
       &        &        &        & \vdots      & \vdots &        & 0      &                 &        &                 &        &                 &        &        \\
0      & 0      &        & 0      & 0  & 0      &        & 1      & h_{n1}^{\prime}    &        &-d_{n}/d_{s}   &        & h_{nn}^{\prime}    & h_{ns}^{\prime}   & r_{n}^{\prime} \\
\end{array}\right|
\end{align*}
その上で,
\begin{align*}
\mathcal{I}_{\mathrm{B}}\leftarrow \mathcal{I}_{\mathrm{B}}/\left\{i_{\mathrm{B}s}\right\}\cup\left\{i_{\mathrm{D}}\right\} \\
\mathcal{I}_{\mathrm{N}}\leftarrow \mathcal{I}_{\mathrm{N}}/\left\{i_{\mathrm{N}s}\right\}\cup\left\{i_{\mathrm{B}s}\right\} \\
\mathcal{I}_{\mathrm{D}}\leftarrow \mathcal{I}_{\mathrm{D}}/\left\{i_{\mathrm{D}}\right\}\cup\left\{i_{\mathrm{B}s}\right\}
\end{align*}
とすれば,形式的に最初と同じ形になります.
これにより新たに非基底となった変数インデックス$i_{\mathrm{B}s}$が$2n+1$ならば,
$q_{i_{\mathrm{B}s}}=v=0$ということになるので,
このときの$\left(\boldsymbol{w},\boldsymbol{z}\right)=\left(\left[q_{1}~\cdots~q_{n}\right]^{\mathrm{T}},\left[q_{n+1}~\cdots~q_{2n}\right]^{\mathrm{T}}\right)$は
$\mathcal{Q}$の元となり,かつ$\boldsymbol{w}^{\mathrm{T}}\boldsymbol{z}=0$を満たす,
すなわちこれが元の相補性問題の解であるということになります.

以上をまとめれば,相補掃き出し法のアルゴリズムは次のようになります.

\begin{algorithm}[tbh]
\caption{\textsc{ComplementaryPivottingMethod}$(\boldsymbol{M},\boldsymbol{p})\rightarrow\boldsymbol{z}$}
\label{alg:direct_divide}
\begin{algorithmic}[1]
\State $s\leftarrow\mathop{\mathrm{arg~min}}_{i}\left\{p_{i}\right\}$
\If{$p_{s}\geq 0$}
  \State {\bf Return} $\boldsymbol{0}$
\EndIf
\State \textsc{InitTableau}$(\boldsymbol{M},\boldsymbol{p})\rightarrow(\mathcal{T},\mathcal{I}_{\mathrm{B}},\mathcal{I}_{\mathrm{N}},\mathcal{I}_{\mathrm{D}})$
\State \textsc{Sweep}$(s;\mathcal{T},\mathcal{I}_{\mathrm{B}},\mathcal{I}_{\mathrm{N}},\mathcal{I}_{\mathrm{D}})$
\State \textsc{SwapIndex}$(s;\mathcal{I}_{\mathrm{B}},\mathcal{I}_{\mathrm{N}},\mathcal{I}_{\mathrm{D}})$
\Repeat
  \State $s\leftarrow$\textsc{FindPivot}$(\mathcal{T},\mathcal{I}_{\mathrm{D}})$
  \If{$s=$nil}
    \State {\bf Return} false
  \EndIf
  \State \textsc{Sweep}$(s;\mathcal{T},\mathcal{I}_{\mathrm{B}},\mathcal{I}_{\mathrm{N}},\mathcal{I}_{\mathrm{D}})$
  \State \textsc{SwapIndex}$(s;\mathcal{I}_{\mathrm{B}},\mathcal{I}_{\mathrm{N}},\mathcal{I}_{\mathrm{D}})$
\Until{$2n+1\notin\mathcal{I}_{\mathrm{N}}$}
\State {\bf Return} \textsc{BuildAnswer}$(\boldsymbol{p},\mathcal{I}_{\mathrm{B}})$
\end{algorithmic}
\end{algorithm}

\textsc{InitTableau}$(\boldsymbol{M},\boldsymbol{p})$はタブローを初期化する処理,
\textsc{Sweep}$(s;\mathcal{T},\mathcal{I}_{\mathrm{B}},\mathcal{I}_{\mathrm{N}},\mathcal{I}_{\mathrm{D}})$はタブローの$s$行目$i_{\mathrm{D}}$列目を軸として掃き出しを行う処理,
\textsc{SwapIndex}$(s;\mathcal{I}_{\mathrm{B}},\mathcal{I}_{\mathrm{N}},\mathcal{I}_{\mathrm{D}})$は基底・非基底・駆動変数インデックスを入れ替える処理,
\textsc{FindPivot}$(\mathcal{T},\mathcal{I}_{\mathrm{D}})$は軸を見つける処理,
\textsc{BuildAnswer}$(\boldsymbol{p},\mathcal{I}_{\mathrm{B}})$は$\boldsymbol{p}$と$\mathcal{I}_{\mathrm{B}}$から解$\boldsymbol{z}$を構成する処理です.



\section{2次計画問題から線形相補性問題への等価変換}

冒頭に述べた通り,凸2次計画問題も双対性に基づいて線形相補性問題に等価変換できます.
これを示しましょう.

次の凸2次計画問題を考えます.
\begin{align*}
\boldsymbol{x}^{*}=\mathop{\mathrm{arg~min}}_{\boldsymbol{x}}\left\{
\left.
z=\frac{1}{2}\boldsymbol{x}^{\mathrm{T}}\boldsymbol{Q}\boldsymbol{x}+\boldsymbol{c}^{\mathrm{T}}\boldsymbol{x}
\right|
\boldsymbol{A}\boldsymbol{x}\geq\boldsymbol{b},
\boldsymbol{x}\geq\boldsymbol{0}
\right\}
\end{align*}
ただし,$\boldsymbol{Q}$は実対称非負定値行列,$\boldsymbol{A}$は定数行列,$\boldsymbol{c}$と$\boldsymbol{b}$はどちらも定数ベクトルです.

まず,$\boldsymbol{x}^{*}$が存在するならば,次を満たす$\boldsymbol{\lambda}$も存在します.
\begin{align*}
\boldsymbol{\mu}^{\mathrm{T}}\boldsymbol{x}^{*}=0, \quad
\boldsymbol{y}^{\mathrm{T}}\boldsymbol{\lambda}=0, \quad
\boldsymbol{\mu}\geq\boldsymbol{0}, \quad
\boldsymbol{y}\geq\boldsymbol{0}, \quad
\boldsymbol{x}^{*}\geq\boldsymbol{0}, \quad
\boldsymbol{\lambda}\geq\boldsymbol{0}
\end{align*}
ただし,
\begin{align*}
\boldsymbol{\mu}\overset{\mathrm{def}}{=}\boldsymbol{Q}\boldsymbol{x}^{*}-\boldsymbol{A}^{\mathrm{T}}\boldsymbol{\lambda}+\boldsymbol{c} \\
\boldsymbol{y}\overset{\mathrm{def}}{=}\boldsymbol{A}\boldsymbol{x}^{*}-\boldsymbol{b} \\
\end{align*}
とおきました.
証明は次の通りです.
$\boldsymbol{A}(\boldsymbol{x}^{*}+\boldsymbol{\varepsilon})\geq\boldsymbol{b}$および$\boldsymbol{x}^{*}+\boldsymbol{\varepsilon}\geq\boldsymbol{0}$を満たす
任意の微小なベクトル$\boldsymbol{\varepsilon}$を考えると,
$z$は$\boldsymbol{x}=\boldsymbol{x}^{*}$において最小となるので,
\begin{align*}
\left.z\right|_{\boldsymbol{x}=\boldsymbol{x}^{*}+\boldsymbol{\varepsilon}}\geq\left.z\right|_{\boldsymbol{x}=\boldsymbol{x}^{*}}
\end{align*}
すなわち
\begin{align*}
\frac{1}{2}\boldsymbol{\varepsilon}^{\mathrm{T}}\boldsymbol{Q}\boldsymbol{\varepsilon}+
(\boldsymbol{c}+\boldsymbol{Q}\boldsymbol{x}^{*})^{\mathrm{T}}\boldsymbol{\varepsilon}\geq 0
\end{align*}
となるはずです.これが満たされるためには
\begin{align*}
(\boldsymbol{c}+\boldsymbol{Q}\boldsymbol{x}^{*})^{\mathrm{T}}\boldsymbol{\varepsilon}\geq 0
\end{align*}
でなければなりません.
ここで,$\boldsymbol{\varepsilon}$についての線形計画問題
\begin{align*}
\boldsymbol{\varepsilon}^{*}=\mathop{\mathrm{arg~min}}_{\boldsymbol{\varepsilon}}\left\{
w=(\boldsymbol{c}+\boldsymbol{Q}\boldsymbol{x}^{*})^{\mathrm{T}}\boldsymbol{\varepsilon}
\left|
\boldsymbol{A}\boldsymbol{\varepsilon}\geq-\boldsymbol{y},
\boldsymbol{\varepsilon}\geq-\boldsymbol{x}^{*}
\right.
\right\}
\end{align*}
を考えると,
$\boldsymbol{\varepsilon}=\boldsymbol{0}$は明らかにこの問題の実行可能解です.
このとき$w=0$ですが,上記より$w\geq 0$ですので,
$\boldsymbol{\varepsilon}=\boldsymbol{0}$はこの問題の最適解となっている,
つまり$\boldsymbol{\varepsilon}^{*}=\boldsymbol{0}$であると分かります.
したがって同時に,上記の線形計画問題の双対問題
\begin{align*}
(\boldsymbol{\lambda},\boldsymbol{\mu})=\mathop{\mathrm{arg~min}}_{(\boldsymbol{\lambda},\boldsymbol{\mu})}\left\{
\boldsymbol{y}^{\mathrm{T}}\boldsymbol{\lambda}+\boldsymbol{x}^{*\mathrm{T}}\boldsymbol{\mu}
\left|
\boldsymbol{A}^{\mathrm{T}}\boldsymbol{\lambda}+\boldsymbol{\mu}=\boldsymbol{c}+\boldsymbol{Q}\boldsymbol{x}^{*},
\boldsymbol{\lambda}\geq\boldsymbol{0},
\boldsymbol{\mu}\geq\boldsymbol{0}
\right.
\right\}
\end{align*}
も最適解を持ち,
しかもそれは$\boldsymbol{y}^{\mathrm{T}}\boldsymbol{\lambda}+\boldsymbol{x}^{*\mathrm{T}}\boldsymbol{\mu}=0$を満たすと分かります.
$\boldsymbol{y}\geq\boldsymbol{0}$,
$\boldsymbol{\lambda}\geq\boldsymbol{0}$,
$\boldsymbol{x}^{*}\geq\boldsymbol{0}$,
$\boldsymbol{\mu}\geq\boldsymbol{0}$ですので,
結局$\boldsymbol{y}^{\mathrm{T}}\boldsymbol{\lambda}=0$かつ$\boldsymbol{x}^{*\mathrm{T}}\boldsymbol{\mu}=0$が成り立つことになります(証明終わり).

逆に,上記を満たす$\boldsymbol{\lambda}$が存在するならば$\boldsymbol{x}^{*}$は元の凸2次計画問題の最適解となります.
証明は次の通りです.
\begin{align*}
z-\left.z\right|_{\boldsymbol{x}=\boldsymbol{x}^{*}}
=
\frac{1}{2}(\boldsymbol{x}-\boldsymbol{x}^{*})^{\mathrm{T}}\boldsymbol{Q}(\boldsymbol{x}-\boldsymbol{x}^{*})
+(\boldsymbol{c}+\boldsymbol{Q}\boldsymbol{x}^{*})^{\mathrm{T}}(\boldsymbol{x}-\boldsymbol{x}^{*})
\end{align*}
であるので,常に
\begin{align*}
z-\left.z\right|_{\boldsymbol{x}=\boldsymbol{x}^{*}}
\geq
(\boldsymbol{c}+\boldsymbol{Q}\boldsymbol{x}^{*})^{\mathrm{T}}(\boldsymbol{x}-\boldsymbol{x}^{*})
\end{align*}
が言えます.
$\boldsymbol{x}$が元の問題の実行可能解ならば,$\boldsymbol{A}\boldsymbol{x}\geq\boldsymbol{b}$,$\boldsymbol{x}\geq\boldsymbol{0}$,$\boldsymbol{\mu}\geq\boldsymbol{0}$,$\boldsymbol{\mu}^{\mathrm{T}}\boldsymbol{x}^{*}=0$かつ
\begin{align*}
\boldsymbol{c}+\boldsymbol{Q}\boldsymbol{x}^{*}=\boldsymbol{A}^{\mathrm{T}}\boldsymbol{\lambda}+\boldsymbol{\mu}
\end{align*}
ですので,
\begin{align*}
z-\left.z\right|_{\boldsymbol{x}=\boldsymbol{x}^{*}}
\geq
(\boldsymbol{A}^{\mathrm{T}}\boldsymbol{\lambda}+\boldsymbol{\mu})^{\mathrm{T}}(\boldsymbol{x}-\boldsymbol{x}^{*})
=\boldsymbol{\lambda}^{\mathrm{T}}\boldsymbol{A}(\boldsymbol{x}-\boldsymbol{x}^{*})
+\boldsymbol{\mu}^{\mathrm{T}}\boldsymbol{x}-\boldsymbol{\mu}^{\mathrm{T}}\boldsymbol{x}^{*}
\\
\geq-\boldsymbol{\lambda}^{\mathrm{T}}(\boldsymbol{A}\boldsymbol{x}^{*}-\boldsymbol{b})=-\boldsymbol{\lambda}^{\mathrm{T}}\boldsymbol{y}=0
\end{align*}
すなわち
\begin{align*}
z\geq\left.z\right|_{\boldsymbol{x}=\boldsymbol{x}^{*}}
\end{align*}
となります(証明終わり).

以上より,
元の凸2次計画問題を解くことは,
\begin{align*}
\boldsymbol{\mu}=\boldsymbol{Q}\boldsymbol{x}^{*}-\boldsymbol{A}^{\mathrm{T}}\boldsymbol{\lambda}+\boldsymbol{c} \\
\boldsymbol{y}=\boldsymbol{A}\boldsymbol{x}^{*}-\boldsymbol{b} \\
\boldsymbol{\mu}^{\mathrm{T}}\boldsymbol{x}^{*}=0, \quad
\boldsymbol{y}^{\mathrm{T}}\boldsymbol{\lambda}=0 \\
\boldsymbol{\mu}\geq\boldsymbol{0}, \quad
\boldsymbol{y}\geq\boldsymbol{0}, \quad
\boldsymbol{x}^{*}\geq\boldsymbol{0}, \quad
\boldsymbol{\lambda}\geq\boldsymbol{0}
\end{align*}
を満たす$\boldsymbol{x}$と$\boldsymbol{\lambda}^{*}$の組を求めることと等価であると分かります.
これは実は
\begin{align*}
\boldsymbol{w}\overset{\mathrm{def}}{=}\begin{bmatrix}
\boldsymbol{\mu} \\ \boldsymbol{y}
\end{bmatrix}, \quad
\boldsymbol{z}\overset{\mathrm{def}}{=}\begin{bmatrix}
\boldsymbol{x}^{*} \\ \boldsymbol{\lambda}
\end{bmatrix}, \quad
\boldsymbol{M}\overset{\mathrm{def}}{=}\begin{bmatrix}
\boldsymbol{Q} & -\boldsymbol{A}^{\mathrm{T}} \\
\boldsymbol{A} & \boldsymbol{O}
\end{bmatrix}, \quad
\boldsymbol{p}\overset{\mathrm{def}}{=}\begin{bmatrix}
\boldsymbol{c} \\ -\boldsymbol{b}
\end{bmatrix}
\end{align*}
とおいたときに,本記事冒頭に示した線形相補性問題になっています.

変数に非負条件$\boldsymbol{x}\geq\boldsymbol{0}$が課されない場合は,
二つの非負ベクトル$\boldsymbol{x}_{+}\geq\boldsymbol{0}$,$\boldsymbol{x}_{-}\geq\boldsymbol{0}$を用いて$\boldsymbol{x}=\boldsymbol{x}_{+}-\boldsymbol{x}_{-}$とおけば良いです.
すなわち
\begin{align*}
\boldsymbol{x}^{*}&=\mathop{\mathrm{arg~min}}_{\boldsymbol{x}}\left\{
\left.
z=\frac{1}{2}(\boldsymbol{x}_{+}-\boldsymbol{x}_{-})^{\mathrm{T}}\boldsymbol{Q}(\boldsymbol{x}_{+}-\boldsymbol{x}_{-})+\boldsymbol{c}^{\mathrm{T}}(\boldsymbol{x}_{+}-\boldsymbol{x}_{-})
\right|
\boldsymbol{A}(\boldsymbol{x}_{+}-\boldsymbol{x}_{-})\geq\boldsymbol{b},
\boldsymbol{x}_{+}\geq\boldsymbol{0},
\boldsymbol{x}_{-}\geq\boldsymbol{0}
\right\}
\\
&=\mathop{\mathrm{arg~min}}_{\boldsymbol{x}}\left\{
\left.
z=\frac{1}{2}\tilde{\boldsymbol{x}}^{\mathrm{T}}\tilde{\boldsymbol{Q}}\tilde{\boldsymbol{x}}+\tilde{\boldsymbol{c}}^{\mathrm{T}}\tilde{\boldsymbol{x}}
\right|
\tilde{\boldsymbol{A}}\tilde{\boldsymbol{x}}\geq\boldsymbol{b},
\tilde{\boldsymbol{x}}\geq\boldsymbol{0}
\right\}
\\
\tilde{\boldsymbol{x}}\overset{\mathrm{def}}{=}\begin{bmatrix} \boldsymbol{x}_{+}^{\mathrm{T}} & \boldsymbol{x}_{-}^{\mathrm{T}}\end{bmatrix}^{\mathrm{T}}
\\
\tilde{\boldsymbol{Q}}\overset{\mathrm{def}}{=}\begin{bmatrix} \boldsymbol{Q} & -\boldsymbol{Q} \\ -\boldsymbol{Q} & \boldsymbol{Q} \end{bmatrix}
\\
\tilde{\boldsymbol{c}}\overset{\mathrm{def}}{=}\begin{bmatrix} \boldsymbol{c}^{\mathrm{T}} & -\boldsymbol{c}^{\mathrm{T}} \end{bmatrix}^{\mathrm{T}}
\\
\tilde{\boldsymbol{A}}\overset{\mathrm{def}}{=}\begin{bmatrix} \boldsymbol{A} & -\boldsymbol{A} \end{bmatrix}
\end{align*}
と変形することで,相補掃き出し法を適用できます.



\end{document}
