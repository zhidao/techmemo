\documentclass[a4paper]{jsarticle}
\usepackage[dvips]{graphicx}
\usepackage{amsmath,amssymb,bm}
\usepackage{ascmac}
\usepackage{algorithm,algpseudocode}
\usepackage{svg}
\usepackage{url}

\flushbottom
\sloppy

\setlength{\paperwidth}{210mm}
\setlength{\paperheight}{297mm}
\setlength{\voffset}{0mm}
\setlength{\hoffset}{0mm}
\setlength{\textwidth}{\paperwidth}
\addtolength{\textwidth}{-30mm}
\setlength{\textheight}{\paperheight}
\addtolength{\textheight}{-60mm}
\setlength{\topmargin}{-1in}
\addtolength{\topmargin}{20mm}
\setlength{\headheight}{0mm}
\setlength{\headsep}{0mm}
\setlength{\footskip}{10mm}
\setlength{\oddsidemargin}{-1in}
\addtolength{\oddsidemargin}{15mm}
\setlength{\columnsep}{7mm}

\title{\bf FFT整理}
\author{\Large{\bf 杉原 知道}}
\date{}

\begin{document}
\maketitle
\vspace{-\baselineskip}

\section{おさらい}

\section{周期関数のフーリエ級数展開}

フーリエ変換の出発点となるのは,「任意の周期関数は調和関数の無限級数で表せる」という命題です.
すなわち,ある関数$f(t)$が周期的境界条件
\begin{align*}
f(t+T)=f(t)\qquad\mbox{for}\quad\forall t\in\mathbb{R},\quad\exist T\in\mathbb{R}
\end{align*}
を満たすとき,次式を満たす数列$\left\{(a_{k},b_{k})\right\}$が存在します.
\begin{align*}
f(t)\equiv\sum_{k=0}^{\infty}\left(a_{k}\cos\frac{2\pi kt}{T} + b_{k}\sin\frac{2\pi kt}{T)\right)
\end{align*}
これが真であるとして,$\left\{(a_{k},b_{k})\right\}$はどうすれば求まるか?が最初に考える問題です.

天下り的ですが,
\begin{align*}
\cos\frac{2\pi jt}{T}\cos\frac{2\pi kt}{T}
\end{align*}






「任意の関数$f(t)$は,正弦関数,余弦関数の重ね合わせで表せる」というのが,フーリエ変換の大前提となる命題です.
\begin{align*}
f(t)\equiv\int_{-\infty}^{\infty}\left\{a(\omega)\cos\omega t+b(\omega)\sin\omega t\right\}\mathrm{d}\omega
\end{align*}
ただし,$a(\omega)$,$b(\omega)$は$\omega$のみに依存する実定数です.

まず,$f(t)$が既知であるときに$a(\omega)$,$b(\omega)$を求めることを考えましょう.
天下り的ですが,$f(t)$に$\cos\omega t$および$\sin\omega t$をそれぞれ乗じると,次のようになります.
\begin{align*}
\cos\omega t f(t)
=\int_{-\infty}^{\infty}\left\{a(\omega)\cos^{2}\omega t+b(\omega)\sin\omega t\cos\omega t\right\}\mathrm{d}\omega
\\
=\int_{-\infty}^{\infty}a(\omega)\cos^{2}\omega t\mathrm{d}\omega
+\int_{-\infty}^{\infty}b(\omega)\sin\omega t\cos\omega t \mathrm{d}\omega
\end{align*}






\end{document}
