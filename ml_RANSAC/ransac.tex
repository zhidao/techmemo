\documentclass[a4paper]{jsarticle}
\usepackage{amsmath,amssymb,bm}
\usepackage{graphicx}

\flushbottom
\sloppy

\setlength{\paperwidth}{210mm}
\setlength{\paperheight}{297mm}
\setlength{\voffset}{0mm}
\setlength{\textwidth}{\paperwidth}
\addtolength{\textwidth}{-30mm}
\setlength{\textheight}{\paperheight}
\addtolength{\textheight}{-50mm}
\setlength{\topmargin}{-1in}
\addtolength{\topmargin}{20mm}
\setlength{\headheight}{0mm}
\setlength{\headsep}{0mm}
\setlength{\footskip}{20mm}
\setlength{\oddsidemargin}{-1in}
\addtolength{\oddsidemargin}{15mm}
\setlength{\columnsep}{7mm}

\title{\Large {\bf RANSAC: RANdom SAmpling Consensus}}
\author{杉原知道}


\begin{document}
%\thispagestyle{empty}

\maketitle

ある未知の代数的システム
$\mathcal{S}=\{\bm{x}|\bm{f}(\bm{x})=\bm{0}\}$
($\bm{x}\in\mathbb{R}^{n}$はシステム変数)
のモデル$\hat{\mathcal{S}}=\{\bm{x}|\hat{\bm{f}}(\bm{x};\bm{\theta})=\bm{0}\}$を,
観測に基づいて同定したい.
$\bm{f}(\bm{x})$のモデルを
システムパラメータ$\bm{\theta}$によって
パラメトリックに表現し,
観測によって得られた標本群$\{\bm{x}_{i}\}$($i=1,\cdots,N$)から
$\bm{\theta}$を同定する問題である.
具体的には,
個々の標本$\bm{x}_{i}$に対する誤差を
\begin{align}
\bm{\varepsilon}_{i}\overset{\mathrm{def}}{=}\hat{\bm{f}}(\bm{x}_{i};\bm{\theta})
\label{eq:sample_error}
\end{align}
と定義し,次の最尤計算により$\bm{\theta}$の同定値$\hat{\bm{\theta}}$を得る.
\begin{align}
\hat{\bm{\theta}}&=\mathop{\mathrm{arg~min}}_{\bm{\theta}}E(\bm{\theta};\{\bm{x}_{i}\})
\label{eq:lsm}
\\
E(\bm{\theta};\{\bm{x}_{i}\})&\overset{\mathrm{def}}{=}\frac{1}{2}\sum_{i=1}^{N}\bm{\varepsilon}_{i}^{\mathrm{T}}\bm{\varepsilon}_{i}
\end{align}
$E(\bm{\theta};\{\bm{x}_{i}\})$が,
$\{\bm{x}_{i}\}$が与えられた下での$\bm{\theta}$の関数であることに注意されたい.


今一つの条件として,
標本には確率$p$でアウトライア(外れ値)が含まれるものとしよう.
この場合,
$N$個全ての標本を用いるとシステム同定精度は低下する.
Random Sampling Consensus(RANSAC)は,
このような場合に用いる方法である.
以下にその流れを説明する.
\begin{enumerate}
\item{
全標本から確率$q$で抽出したある小標本群を$\mathcal{X}_{k}$とする.
}
\item{
$\mathcal{X}_{k}$からシステムパラメータを同定し$\hat{\bm{\theta}}_{k}$とする.
\begin{align}
\hat{\bm{\theta}}_{k}=\mathop{\mathrm{arg~min}}_{\bm{\theta}}E(\bm{\theta};\mathcal{X}_{k})
\label{eq:lsm_sample}
\end{align}
}
\item{
全標本について,
$\hat{\bm{\theta}}_{k}$に対する誤差
\begin{align}
\bm{\varepsilon}_{ki}\equiv\hat{\bm{f}}(\bm{x}_{i};\hat{\bm{\theta}}_{k})
\end{align}
を求め,
ある閾値$e_{\mathrm{th}}$に対し$\bm{\varepsilon}_{ki}^{\mathrm{T}}\bm{\varepsilon}_{ki}<e_{\mathrm{th}}^{2}$を満たす標本の数$n_{k}$をカウントする.
}
\item{
上記の処理を$n$回行い,$n_{k}$が最大となる$\hat{\bm{\theta}}_{k}$を$\hat{\bm{\theta}}$の一次値として採択する.
}
\item{
全標本について,
$\hat{\bm{\theta}}$に対する誤差
\begin{align}
\bm{\varepsilon}_{i}\equiv\hat{\bm{f}}(\bm{x}_{i};\hat{\bm{\theta}})
\end{align}
を求め,
$\bm{\varepsilon}_{i}^{\mathrm{T}}\bm{\varepsilon}_{i}<e_{\mathrm{th}}^{2}$を満たす
$\bm{x}_{i}$のセットを用いて$\hat{\bm{\theta}}$をリファインする.
}
\end{enumerate}
なお,
オリジナルの方法では最後にモデルリファインするプロセスは無い.

\begin{quote}
出典:
Martin A. Fischler and Robert C. Bolles,
``Random Sample Consensus: A Paradigm for Model Fitting with Applications to Image Analysis and Automated Cartography''
Communications of the ACM. Vol, 24, No. 6, pp. 381--395, 1981.
\end{quote}

\end{document}
