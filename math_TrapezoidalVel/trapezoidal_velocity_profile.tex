\documentclass[a4paper]{jsarticle}
\usepackage[dvips]{graphicx}
\usepackage{svg}
\usepackage{amsmath,amssymb,bm}

\everymath{\displaystyle}

\flushbottom
\sloppy

\setlength{\paperwidth}{210mm}
\setlength{\paperheight}{297mm}
\setlength{\voffset}{0mm}
\setlength{\hoffset}{0mm}
\setlength{\textwidth}{\paperwidth}
\addtolength{\textwidth}{-30mm}
\setlength{\textheight}{\paperheight}
\addtolength{\textheight}{-60mm}
\setlength{\topmargin}{-1in}
\addtolength{\topmargin}{20mm}
\setlength{\headheight}{0mm}
\setlength{\headsep}{0mm}
\setlength{\footskip}{10mm}
\setlength{\oddsidemargin}{-1in}
\addtolength{\oddsidemargin}{15mm}
\setlength{\columnsep}{7mm}

\title{\bf 台形型速度カーブを用いた移動}
\author{\Large{\bf 杉原 知道}}
\date{}

\begin{document}
\maketitle
\vspace{-\baselineskip}

\section{最大速度が与えられている場合に移動時間を求める}

\begin{figure}[th]
\begin{center}
\includesvg[width=.5\textwidth]{velocity_profile_trapezoidal.svg}
\end{center}
\end{figure}

初期速度$v_{1}$から時刻$T_{1}$まで加速度$a$($>0$)で加速して$v_{\mathrm{M}}$に至り,
その後の時刻$T_{2}$に加速度$-a$で減速を開始して時刻$T$に終端速度$v_{2}$に至る,
かつその間の移動距離が$D$となるよう速度変化させることにする.
このときの時刻$T_{1}$,$T_{2}$および$T$を求めたい.
ただし,他の変数は全て与えられるものとする.

まず,速度の条件
\begin{align}
v_{1}+aT_{1}&=v_{\mathrm{M}} \\
v_{\mathrm{M}}-a(T-T_{2})&=v_{2}
\end{align}
が成り立つので,
\begin{align}
T_{1}&=\frac{v_{\mathrm{M}}-v_{1}}{a}
\label{eq:t1_from_vel}
\\
T-T_{2}&=\frac{v_{\mathrm{M}}-v_{2}}{a}
\label{eq:t2_from_vel}
\end{align}
を得る.
時刻$t=0\sim T_{1}$での移動距離$d_{1}$は
\begin{align}
d_{1}=v_{1}T_{1}+\frac{1}{2}aT_{1}^{2}
\label{eq:d1}
\end{align}
であり,時刻$t=T_{2}\sim T$での移動距離$d_{3}$は
\begin{align}
d_{3}=v_{2}(T-T_{2})+\frac{1}{2}a(T-T_{2})^{2}
\label{eq:d3}
\end{align}
である.
速度が一定値$v_{\mathrm{M}}$である時刻$t=T_{1}\sim T_{2}$での移動距離を考えると
\begin{align}
D=d_{1}+d_{3}+v_{\mathrm{M}}(T_{2}-T_{1})
\label{eq:total_distance}
\end{align}
が成り立つ,すなわち
\begin{align}
T_{2}-T_{1}=\frac{D-d_{1}-d_{3}}{v_{\mathrm{M}}}
\end{align}
である.
よって,
\begin{align}
T_{2}&=T_{1}+\frac{D-d_{1}-d_{3}}{v_{\mathrm{M}}}
\\
T&=T_{2}+\frac{v_{\mathrm{M}}-v_{2}}{a}
\end{align}
となる.
時刻に対する移動距離$d(t)$は
\begin{align}
d(t)=\begin{cases}
v_{1}t+\frac{1}{2}at^{2} & (0\leq t<T_{1})
\\
v_{1}T_{1}+\frac{1}{2}aT_{1}^{2}+v_{\mathrm{M}}(t-T_{1}) & (T_{1}\leq t<T_{2})
\\
D-\frac{1}{2}a(T-t)^{2} & (T_{2}\leq t\leq T)
\end{cases}
\end{align}
となる.




\begin{figure}[h]
\begin{center}
\includesvg[width=.3\textwidth]{velocity_profile_nonflat.svg}
\end{center}
\end{figure}

上記においてもし$T_{2}<T_{1}$となっているならば,
速度が$v_{\mathrm{M}}$に達することなく距離$D$移動できるということを意味する.
この場合,
初期速度$v_{1}$から時刻$T^{\prime}$まで加速度$a$($>0$)で加速し,
その直後に加速度$-a$に転じて時刻$T$まで減速して終端速度$v_{2}$に至る,
かつその間の移動距離が$D$となるように速度変化させることを考える.
このときの$T^{\prime}$および$T$を求めよう.

時刻$T^{\prime}$における速度を$v_{\mathrm{M}}^{\prime}$とおくと,速度の条件
\begin{align}
v_{1}+aT^{\prime}&=v_{\mathrm{M}}^{\prime} \\
v_{\mathrm{M}}^{\prime}-a(T-T^{\prime})&=v_{2}
\end{align}
が成り立つので,
\begin{align}
T^{\prime}&=\frac{v_{\mathrm{M}}^{\prime}-v_{1}}{a} \\
T-T^{\prime}&=\frac{v_{\mathrm{M}}^{\prime}-v_{2}}{a}
\end{align}
を得る.
時刻$t=0\sim T^{\prime}$での移動距離$d_{1}^{\prime}$は
\begin{align}
d_{1}^{\prime}=v_{1}T^{\prime}+\frac{1}{2}aT^{\prime 2}
\end{align}
であり,時刻$t=T^{\prime}\sim T$での移動距離$d_{2}^{\prime}$は
\begin{align}
d_{2}^{\prime}=v_{2}(T-T^{\prime})+\frac{1}{2}a(T-T^{\prime})^{2}
\end{align}
である.
移動距離について
\begin{align}
D=d_{1}^{\prime}+d_{2}^{\prime}
\end{align}
が成り立つ,すなわち
\begin{align}
&v_{1}T^{\prime}+\frac{1}{2}aT^{\prime 2}
+v_{2}(T-T^{\prime})+\frac{1}{2}a(T-T^{\prime})^{2}
=D
\nonumber \\
\Leftrightarrow\qquad&
v_{\mathrm{M}}^{\prime}=\sqrt{\frac{v_{1}^{2}+v_{2}^{2}+2aD}{2}}
\end{align}
である.
これを上式に代入すれば,$T^{\prime}$および$T$が順次得られる.
$T_{1}=T_{2}=T^{\prime}$とすれば,
時刻に対する移動距離$d(t)$は先のものと同一になる.


%% \section{移動時間が与えられている場合に最大速度を求める}

%% 前節と同じく,
%% 初期速度$v_{1}$から時刻$T_{1}$まで加速度$a$($>0$)で加速して$v_{\mathrm{M}}$に至り,
%% その後の時刻$T_{2}$に加速度$-a$で減速を開始して時刻$T$に終端速度$v_{2}$に至る,
%% かつその間の移動距離が$D$となるよう速度変化させることにする.
%% ただし,今度は時刻$T_{1}$,$T_{2}$および最大速度$v_{\mathrm{M}}$を求めたい.
%% $T$を含め,他の変数は全て与えられるものとする.
%% 速度の条件
%% \begin{align}
%% T_{1}&=\frac{v_{\mathrm{M}}-v_{1}}{a}
%% \tag{\ref{eq:t1_from_vel}}
%%  \\
%% T-T_{2}&=\frac{v_{\mathrm{M}}-v_{2}}{a}
%% \tag{\ref{eq:t2_from_vel}}
%% \end{align}
%% と移動距離の条件
%% \begin{align}
%% \frac{1}{2}(v_{1}+v_{\mathrm{M}})T_{1}
%% +v_{\mathrm{M}}(T_{2}-T_{1})
%% +\frac{1}{2}(v_{2}+v_{\mathrm{M}})(T-T_{2})
%% =D
%% \end{align}
%% が同様に成り立つ.
%% これらより$T_{1}$,$T_{2}$を消去すると
%% \begin{align}
%% v_{\mathrm{M}}^{2}-2\cdot\frac{aT+v_{1}+v_{2}}{2}v_{\mathrm{M}}+\frac{2aD+v_{1}^{2}+v_{2}^{2}}{2}=0
%% \end{align}
%% となる.
%% これを$v_{\mathrm{M}}(>0)$について解けば,
%% \begin{align}
%% v_{\mathrm{M}}
%% =
%% \frac{aT+v_{1}+v_{2}}{2}
%% +\sqrt{
%% \left(\frac{aT+v_{1}+v_{2}}{2}\right)^{2}-\frac{2aD+v_{1}^{2}+v_{2}^{2}}{2}
%% }
%% \end{align}
%% を得る.
%% $v_{\mathrm{M}}$が虚数となるときは,
%% 時刻$T$までに距離$D$は移動できないということを意味する.


%% \begin{align}
%% v_{\mathrm{M}}=\frac{aT+v_{1}+v_{2}}{2}
%% \end{align}

%% \begin{align}
%% T_{1}=T_{2}=\frac{aT-v_{1}+v_{2}}{2a}
%% \end{align}

%% \begin{align}
%% (v_{1}-v_{2})T_{1}
%% +(v_{2}+v_{\mathrm{M}})T
%% =2D
%% \end{align}





\end{document}
